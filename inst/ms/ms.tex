\emph{Other title ideas:}

\begin{itemize}
\item
  Portfolio optimization reveals\ldots{}
\item
  something more general with animals instead of just salmon?
\item
  Portfolio theory reveals conservation strategies that\ldots{}
\item
  Intelligent tinkering with ecological portfolios: patterns of
  environmental response diversity drive salmon conservation priorities
\item
  Keeping every cog and wheel\ldots{}
\end{itemize}

\section{Abstract}

\begin{enumerate}
\def\labelenumi{\arabic{enumi}.}
\item
  Managing risk is fundamental to the conservation of an endangered
  species. When an endangered species exists as a metapopulation, we can
  manage risk at two levels: at the population level or at the
  metapopulation level. Whereas risk is typically managed at the
  population level, a portfolio approach to managing risk might consider
  how conservation affects the ``weight'' of each population in a
  metapopulation ``portfolio''.
\item
  Here, we ask how a portfolio approach to managing risk can inform the
  spatial conservation of metapopulations in a changing world. To answer
  this, we develop a salmon metapopulation simulation in which
  population-specific productivity is driven by spatially-distributed
  environmental tolerance and patterns of short- and long-term
  environmental change. We then implement different spatial conservation
  ``rules of thumb'' that control the population-specific carrying
  capacities and evaluate the salmon portfolios along risk and return
  axes, similarly to how financial portfolios are assessed.
\item
  Our results show, first, that maintaining populations with a variety
  of environmental tolerances gives the best chance at an efficient
  ecological portfolio --- minimizing metapopulation variance while
  maximizing metapopulation growth rate. This finding emphasizes the
  risk of allowing large spatial blocks of habitat destruction, say
  through the development of dams. Second, we show that focusing on
  well-performing stocks now at the detriment of others is at best
  equivalent to a risky but efficient portfolio and is more likely a
  risky and inefficient portfolio --- it neither minimizes
  metapopulation variance nor maximizes growth rate compared to other
  strategies. Third, we show that maintaining more populations reduces
  metapopulation risk for the same spatial conservation strategy. Given
  a lack of knowledge of how populations respond to the environment, the
  most risk-averse approach is to conserve as many populations as
  possible.
\item
  Our findings highlight three key points: (1) the conservation priority
  of maintaining biocomplexity and therefore environmental response
  diversity,
\end{enumerate}

\begin{enumerate}
\def\labelenumi{(\arabic{enumi})}
\setcounter{enumi}{1}
\itemsep1pt\parskip0pt\parsep0pt
\item
  the research priority of identifying differences in environmental
  tolerance given predicted environmental changes, and (3) the utility
  of considering risk for groups of fish stocks --- especially given
  environmental, biological, and implementation uncertainty --- through
  the lens of portfolio theory.
\end{enumerate}

\section{Introduction}

Managing risk fundamental to conservation of metapopulations

Portfolios provide another way of managing risk - explain mean-variance
optimization and how this translates to ecology \citep{Figge2004}
\citep{Hoekstra2012} \citep[\citet{Ando2012}]{Ando2011}
\citep[\citet{Markowitz1959}]{Markowitz1952}

We use salmon as an example, describe the components of the example,
provide references on metapopulation existence \citep{Schindler2010}

Research questions, and briefly our approach - Monte Carlo simulation
modelling, as simple a model as possible while retaining important
realistic aspects of salmon metapopulations

\section{Methods}

\subsection{Simulation model components}

general overview

See Figure \ref{f:sim-flow} for an illustration of the simulation
structure.

\subsection{Salmon metapopulation dynamics}

We are using a Ricker curve,

\begin{equation}
R_{ti} = S_{ti}e^{a_{ti}(1-S_{ti}/b_i) + w_{ti}},
\end{equation}

\noindent
where $t$ represents a generation time, $i$ represents a population, $R$
is the number of returns, $S$ is the number of spawners, $a$ is the
productivity parameter (which can vary with the environmental signal),
and $b$ is the density-dependent term (which is used as the asset
weights in the portfolios). The term $w_{ti}$ represents first-order
autocorrelated error (AR1). Formally,
$w_{ti} = w_{ti-1} \rho_w + v_{ti}$, where $v_{ti}$ represents
independent and normally distributed error with mean 0 and standard
deviation of $\sigma_v$. The parameter $\rho_w$ represents the
correlation between residuals from subsequent generations.

@cooper1999

We implemented straying as in \citet{Cooper1999}. We generate a matrix
that represents the fraction of straying between any two populations. We
have set up the metapopulation (thus far) in a very simple scenario: the
populations are arranged in a line and those that are nearer to each
other are more likely to stray between each other {[}insert caveats{]}.
Two parameters control the straying: the fraction of fish $f_{stray}$
that stray from their natal stream in any given generation and the rate
at which this straying between streams decays with distance $m$. We
calculated the number of salmon straying from stream $j$ to stream $i$
as,

\begin{equation}
  \mathrm{strays}_{tij} = f_{stray} R_{tj}
    \frac{e^{-m \lvert i-j \rvert }}
      {\displaystyle\sum\limits_{\substack{k = 1 \\
    k \neq j}}^{n} e^{-m \lvert k-j \rvert }},
  \label{eq:stray}
\end{equation}

\noindent
where $R_{tj}$ is the number of returning salmon at generation $t$ whose
natal stream was stream $j$. The subscript $k$ represents a stream ID
and $n$ the number of populations. The denominator is a normalizing
constant to ensure the desired fraction of fish stray. See Figure
\ref{f:stray} for an example straying matrix.

\subsection{Environmental dynamics}

\subsection{Environmental response diversity}

Scenarios? More variable are also more productive?\ldots{}

\subsection{Portfolio optimization}

\subsection{Conservation rules of thumb}

\section{Results}

Fig. \ref{f:eg-n-linear}

Fig. \ref{f:eg-n-arma}

Fig. \ref{f:eg-sp-arma}

Fig. \ref{f:eg-sp-linear}

\subsection{Base case}

\subsection{Effect of X on \ldots{}}

\subsection{Effect of Y on \ldots{}}

\section{Discussion}

\section{Acknowledgements}

\section{References}

\bibliographystyle{apalike}

\bibliography{jshort,ms}

\clearpage

\section{Tables}

\begin{table}[h!]
\centering
\small
\caption{Components of salmon metapopulation portfolios}
\begin{tabular}{p{3.6cm}p{7.5cm}}
\toprule
Component & Definition for the salmon portfolio\\
\midrule
Assets             & Stream-level salmon populations; possibly a Viable
                     Salmonid Population\\
Portfolio          & The salmon metapopulation; possibly an Evolutionarily
                     Significant Unit\\
Portfolio managers & Salmon managers\\
Investors          & Salmon managers, conservation agency, or salmon fishers\\
Asset weights      & Carrying capacity (specifically the $b$ parameter in
                     a Ricker model)\\
Asset returns      & Rate of change of generation-to-generation salmon
                     metapopulation abundance\\
Asset risk         & Variance of generation-to-generation salmon metapopulation
                     abundance\\
\bottomrule
\end{tabular}
\label{tab:port-components}
\end{table}

\clearpage

\begin{table}[h!]
\centering
\small
\caption{Salmon metapopulation parameters with base case and alternate values.}
\begin{tabular}{p{7.0cm}p{1.6cm}p{3.2cm}}
\toprule
Description                                                      & Parameter       & Base [lower, upper] \\
\midrule
Stock-recruit residual standard deviation (on log scale)         & $\sigma_v$      & 0.30 [0.05, 0.50] \\
First order (AR1) serial correlation of stock-recruit residuals  & $\rho_w$        & 0.40 [0, 0.80] \\
Fraction of fish that stray from natal streams                   & $f_{stray}$     & 0.02 [0, 0.10] \\
Exponential rate of decay of straying with distance              & $m$             & 0.3 [0.05, 0.5] \\
Standard deviation of beta distribution for implementation error & $\sigma_{impl}$ & 0.05 [0, 0.20] \\
Frequency of assessment (years)                                  & $f_{assess}$    & 20 [5, 50] \\
\bottomrule
\end{tabular}
\label{tab:salm-pars}
\end{table}

\clearpage

\section{Figures}

`

\begin{figure}[htbp]
\centering
\includegraphics[height=5.5in]{../examples/simulation-diagram2.pdf}
\caption{Flow chart of the salmon-metapopulation simulation. There are $n$ 
salmon populations and $t$ generations. Blue text indicates values that are 
generated before the simulation progresses through time. Red text indicates 
steps in which calculations are performed. Black text indicates values that are 
calculated. Grey text indicates parameters that can be set. Green text 
indicates the looping structure of the simulation.}
\label{f:sim-flow}
\end{figure}

`

\begin{figure}[htbp]
\centering
\includegraphics[width=4.0in]{../examples/spatial-arma-sim.pdf}
\caption{The components of an example metapopulation simulation.  We show, from 
top to bottom, the environmental signal, the resulting productivity parameter 
(Ricker $a$), the salmon returns, fisheries catch, salmon escapement, salmon 
straying from their natal streams, salmon joining from other streams, 
stock-recruit residuals on a log scale, and the estimated $a$ and $b$ 
parameters in the fitted Ricker curve. The colored lines indicate populations 
that thrive at low (cool colours) to high (warm colours) environmental values.}
\label{f:sp-eg}
\end{figure}

\begin{figure}[htbp]
\centering
\includegraphics[width=3.0in]{../examples/thermal-curve-scenarios.pdf}
\caption{Different ways of prioritizing response-diversity conservation. The 
coloured lines represent environmental-tolerance curves for conserved 
populations. The colours identify populations that thrive at low (cool colours) 
to high (warm colours).}
\label{f:curves}
\end{figure}

\begin{figure}[htbp]
\centering
\includegraphics[width=5.0in]{../examples/spatial-mv.pdf}
\caption{The importance of preserving environmental response diversity through 
spatial conservation strategies. The conservation strategies correspond to 
Fig.~\ref{f:curves} and represent conserving a range of responses (green), the 
most stable populations only (orange), or one type of environmental response 
(purple and pink).  In risk-return space we show environmental scenarios that 
are comprised primarily of (a) short-term and (b) long-term environmental 
fluctuations (see Fig.~X). The dots show simulated metapopulations and the 
contours show 25\% and 75\% quantiles across 500 simulations per strategy. We 
also show example metapopulation abundance time series for the (c, e) 
short-term and (d, f) long-term  environmental-fluctuation scenarios.}
\label{f:sp-mv}
\end{figure}

\begin{figure}[htbp]
\centering
\includegraphics[width=5.0in]{../examples/cons-plans-n.pdf}
\caption{The importance of preserving as many salmon populations as possible 
when we don't know how response diversity is distributed. In risk-return space 
we show environmental scenarios that are comprised primarily of (a) short-term 
and (b) long-term environmental fluctuations (see Fig.~X). We show 
metapopulations in which 2 (red), 4 (orange), 8 (yellow), or 16 (green) 
populations of random response diversity are conserved. The dots show simulated 
metapopulations and the contours show 25\% and 75\% quantiles across 500 
simulations per strategy. We also show example metapopulation (c) 
rate-of-change and (d) abundance time series for a the short-term 
environmental-fluctuation scenario.}
\label{f:n-mv}
\end{figure}
