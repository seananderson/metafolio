\emph{Other title ideas:}

\begin{itemize}
\item
  Portfolio optimization reveals\ldots{}
\item
  something more general with animals instead of just salmon?
\item
  Portfolio theory reveals conservation strategies that\ldots{}
\item
  Intelligent tinkering with ecological portfolios: patterns of
  environmental response diversity drive salmon conservation priorities
\item
  Keeping every cog and wheel\ldots{}
\end{itemize}

\section{Abstract}

\begin{enumerate}
\def\labelenumi{\arabic{enumi}.}
\item
  Managing risk is fundamental to the conservation of an endangered
  species. When an endangered species exists as a metapopulation, we can
  manage risk at two levels: at the population level or at the
  metapopulation level. Whereas risk is typically managed at the
  population level, a portfolio approach to managing risk might consider
  how conservation affects the ``weight'' of each population in a
  metapopulation ``portfolio''.
\item
  Here, we ask how a portfolio approach to managing risk can inform the
  spatial conservation of metapopulations in a changing world. To answer
  this, we develop a salmon metapopulation simulation in which
  population-specific productivity is driven by spatially-distributed
  environmental tolerance and patterns of short- and long-term
  environmental change. We then implement different spatial conservation
  ``rules of thumb'' that control the population-specific carrying
  capacities and evaluate the salmon portfolios along risk and return
  axes, similarly to how financial portfolios are assessed.
\item
  Our results show, first, that maintaining populations with a variety
  of environmental tolerances gives the best chance at an efficient
  ecological portfolio --- minimizing metapopulation variance while
  maximizing metapopulation growth rate. This finding emphasizes the
  risk of allowing large spatial blocks of habitat destruction, say
  through the development of dams. Second, we show that focusing on
  well-performing stocks now at the detriment of others is at best
  equivalent to a risky but efficient portfolio and is more likely a
  risky and inefficient portfolio --- it neither minimizes
  metapopulation variance nor maximizes growth rate compared to other
  strategies. Third, we show that maintaining more populations reduces
  metapopulation risk for the same spatial conservation strategy. Given
  a lack of knowledge of how populations respond to the environment, the
  most risk-averse approach is to conserve as many populations as
  possible.
\item
  Our findings highlight three key points: (1) the conservation priority
  of maintaining biocomplexity and therefore environmental response
  diversity, (2) the research priority of identifying differences in
  environmental tolerance given predicted environmental changes, and (3)
  the utility of considering risk for groups of fish stocks ---
  especially given environmental, biological, and implementation
  uncertainty --- through the lens of portfolio theory.
\end{enumerate}

\section{Introduction}

Managing risk is fundamental to the conservation of an endangered
species. When an endangered species exists as a metapopulation, we can
manage risk at two levels: at the population level or at the
metapopulation level. Typically we treat sources of risk at the
metapopulation level as external and uncontrollable and so we manage
risk by altering fishing or hunting on a population level as well as
improving the connectivity of populations.

The management of financial portfolios provides another way of
considering risk. Economists consider the risk and performance of a
financial portfolio as a function of the weighting of individual assets
that make up the portfolio. Modern Portfolio Theory (MPT) proposes that
there is a set of portfolios that maximizes expected return for a level
of expected risk or minimizes expected risk for a level of expected
return \citep{Markowitz1952, Markowitz1959}. This optimal set contains
portfolios that range along a continuum of risk-tolerance; economists
refer to this set as the efficient frontier.

Portfolios provide another way of managing risk - explain mean-variance
optimization and how this translates to ecology
\citep{Figge2004}\citep{Hoekstra2012}
\citep[\citet{Ando2012}]{Ando2011}\citep[\citet{Markowitz1959}]{Markowitz1952}

We use salmon as an example, describe the components of the example,
provide references on metapopulation existence \citep{Schindler2010}

Research questions, and briefly our approach - Monte Carlo simulation
modelling, as simple a model as possible while retaining important
realistic aspects of salmon metapopulations

Here, we ask how a portfolio approach to managing risk can inform the
spatial conservation of metapopulations in a changing world. To answer
this, we develop a salmon metapopulation simulation in which
population-specific productivity is driven by spatially-distributed
environmental tolerance and patterns of short- and long-term
environmental change. We then implement different conservation ``rules
of thumb'' that control the population-specific carrying capacities and
evaluate the salmon portfolios along risk and return axes, similarly to
how financial portfolios are assessed. We show\ldots{} (briefly)

\section{Methods}

\subsection{Overview}

We developed a salmon metapopulation simulation model that includes a
stock-recruit relationship, demographic stochasticity, straying between
populations, varying responses to the environment, escapement target
setting, and implementation uncertainty. Under two kinds of
environmental regimes we tested different conservation rules of thumb
and evaluated these plans in risk-return space similar to how financial
managers evaluate financial portfolios. See Figure \ref{f:sim-flow} for
an illustration of the overall simulation structure.

\subsection{Defining the ecological portfolio}

In our ecological portfolio, we defined assets as stream-level
populations and the portfolio as a salmon metapopulation (Table
\ref{t:port}). The portfolio investors were the stakeholders in the
fishery and metapopulation performance. For example, we could consider
fisheries managers, conservation agencies, or first Nations groups as
investors. We defined asset value as abundance (returns) of stream
populations and value of the portfolio as the overall metapopulation
abundance. In this scenario, the equivalent to financial rate of return
is the generation-to-generation rate of change of metapopulation
abundance. We defined the financial asset investment weights as the
carrying capacity of the stream populations, specifically the unfinished
equilibrium stock size.

\subsection{Salmon metapopulation dynamics}

The salmon metapopulation dynamics in our simulation were governed by a
spawner-return relationship with stochastic noise and straying between
populations.

\subsubsection{Spawner-return relationship}

We defined the spawner-return relationship with a Ricker model
\citep{Ricker1954},

\begin{equation}
R_{ti} = S_{ti}e^{a_{ti}(1-S_{ti}/b_i) + w_{ti}},
\end{equation}

\noindent where $t$ represents a generation time, $i$ represents a
population, $R$ is the number of returns, $S$ is the number of spawners,
$a$ is the productivity parameter (which can vary with the environment),
and $b$ is the density-dependent term (which is used as the asset
weights in the portfolios). The term $w_{ti}$ represents first-order
autocorrelated error. Formally, $w_{ti} = w_{ti-1} \rho_w + v_{ti}$,
where $v_{ti}$ represents independent and normally distributed error
with mean 0 and standard deviation of $\sigma_v$. The parameter $\rho_w$
represents the correlation between residuals from subsequent
generations.

\subsubsection{Straying}

We implemented straying as in \citet{Cooper1999}. We set up the
metapopulation in a simple scenario: the populations were arranged in a
line and those that were nearer to each other were more likely to stray
between each other. Two parameters controlled the straying: the fraction
of fish $f_{\mathrm{stray}}$ that stray from their natal stream in any
generation and the rate $m$ at which this straying between streams
decays with distance. We calculated the number of salmon straying from
stream $j$ to stream $i$ as,

\begin{equation}
  \mathrm{strays}_{tij} = f_{\mathrm{stray}} R_{tj}
    \frac{e^{-m \lvert i-j \rvert }}
      {\displaystyle\sum\limits_{\substack{k = 1 \\
    k \neq j}}^{n} e^{-m \lvert k-j \rvert }},
  \label{eq:stray}
\end{equation}

\noindent
where $R_{tj}$ is the number of returning salmon at generation $t$ whose
natal stream was stream $j$. The subscript $k$ represents a stream ID
and $n$ the number of populations. The denominator is a normalizing
constant to ensure the desired fraction of fish stray. See Figure
\ref{f:stray} for an example straying matrix.

\subsection{Fishing}

Our simulation used a simple set of rules to establish escapement
targets and implement harvesting. Every 5(TODO CAN VARY) years our
simulation fitted a spawner-return function and target harvest rate
$H_{\mathrm{tar}}$ was set based on \citet{Hilborn1992} as

\begin{equation}
  H_{\mathrm{tar}} = \frac{A}{b (0.5 - 0.07a)}
  \label{eq:esc}
\end{equation}

\noindent
where $A$ represents the return abundance and $a$ and $b$ represent the
Ricker model parameters. We included implementation uncertainty in the
actual harvest rate $H_{\mathrm{act}}$ as

\begin{equation}
  H_{\mathrm{act}} = \mathrm{beta}(\alpha_h, \beta_h)
\end{equation}

\noindent
where $\alpha_h$ and $\beta_h$ are the location and shape parameters in
a beta distribution. They can be calculated from the desired mean
$\mu_h$ and standard deviation $\sigma_h$ as: TODO ADD REFERENCES

\begin{align}
  \alpha_h &= \frac{\mu_h^2(1 - \mu_h)}{\sigma_h - 1/\mu_h}\\
   \beta_h &= \frac{\alpha}{\mu_h - 1}.
\end{align}

\noindent
Further, to establish a range of spawner-return values and to mimic the
start of an open-access fishery, for the first 20 (TODO variable) years
we drew the fraction of fish harvested randomly from a uniform
distribution between 0.1 and 0.9. TODO CONSIDER BETTER RANGE HERE

\subsection{Environmental dynamics}

We can generally break environmental dynamics down into short-term and
long-term changes. We evaluated the contribution of the environment to
metapopulation performance under these two components separately.

short-term forcing: stationary AR1; long-term forcing: linear increase

\subsection{Conservation rules of thumb}

We evaluated two sets of conservation rules of thumb: (1) spatial
response diversity rules of thumb in an idealistic situation where you
can detect response diversity, and (2) a more realistic scenario where
we know little about response diversity and we're left with a choice of
how many populations to conserve. We then considered how optimal these
strategies are across all possible conservation-prioritization
strategies.

\subsubsection{Spatial conservation strategies}

We evaluated four spatial conservation rules of thumb (Figure
\ref{f:curves}). In all scenarios, four populations are conserved and
the unfished equilibrium biomass of the remaining populations is reduced
to near elimination (five salmon). These reduced populations can
therefore still receive straying salmon but are unlikely to re-build on
their own to any substantial volume of salmon.

The four scenarios are:

\begin{enumerate}
\def\labelenumi{\arabic{enumi}.}
\item
  Conserve an even sampling of response diversity. In other words,
  conserve alternating subpopulations.
\item
  Conserve the most stable populations only.
\item
  Conserve one half of the metapopulation.
\item
  Conserve the other half of the metapopulation.
\end{enumerate}

\subsubsection{How many populations to conserve?}

In reality we rarely know precise levels of response diversity. We
therefore additionally considered a case where the conservation is
randomly assigned with respect to response diversity but where different
numbers of streams can be conserved. We considered conserving from two
to 16 streams. Similarly to the spatial strategies, we reduced the
carrying capacity of the remaining streams to the nominal level of five
salmon.

\subsubsection{How optimal are the conservation strategies?}

To test how optimal the strategies are, we generated metapopulation
portfolios in which the investment weights (Ricker $b$ parameters) were
randomly assigned. This Monte-Carlo strategy illustrates a large sample
of possible portfolios in risk-return space. We then placed the
conservation strategies listed above into this space to determine how
close to the efficient frontier they lie.

\section{Results}

Fig. \ref{f:eg-n-linear}

Fig. \ref{f:eg-n-arma}

Fig. \ref{f:eg-sp-arma}

Fig. \ref{f:eg-sp-linear}

\subsection{Base case}

\subsection{Effect of X on \ldots{}}

\subsection{Effect of Y on \ldots{}}

\section{Discussion}

\section{Acknowledgements}

\section{References}

\bibliographystyle{apalike}

\bibliography{jshort,ms}

\clearpage

\section{Tables}

\begin{table}[h!]
\centering
\small
\caption{Components of salmon metapopulation portfolios}
\begin{tabular}{p{3.6cm}p{7.5cm}}
\toprule
Component          & Definition for the salmon portfolio\\
\midrule
Assets             & Stream-level salmon populations; possibly a Viable Salmonid Population\\
Portfolio          & The salmon metapopulation; possibly an Evolutionarily Significant Unit\\
Portfolio managers & Salmon managers\\
Investors          & Salmon managers, conservation agency, or salmon fishers\\
Asset weights      & Carrying capacity (specifically the $b$ parameters in a Ricker model)\\
Asset returns      & Rate of change of generation-to-generation salmon metapopulation abundance\\
Asset risk         & Variance of generation-to-generation salmon metapopulation abundance\\
\bottomrule
\end{tabular}
\label{t:port}
\end{table}

\clearpage

\begin{table}[h!]
\centering
\small
\caption{Salmon metapopulation parameters with base case and alternate values.}
\begin{tabular}{p{7.0cm}p{1.6cm}p{3.2cm}}
\toprule
Description                                                      & Parameter             & Base [lower, upper] \\
\midrule
Stock-recruit residual standard deviation (on log scale)         & $\sigma_v$            & 0.30 [0.05, 0.50] \\
First order (AR1) serial correlation of stock-recruit residuals  & $\rho_w$              & 0.40 [0, 0.80] \\
Fraction of fish that stray from natal streams                   & $f_{\mathrm{stray}}$  & 0.02 [0, 0.10] \\
Exponential rate of decay of straying with distance              & $m$                   & 0.3 [0.05, 0.5] \\
Standard deviation of beta distribution for implementation error & $\sigma_{h}$          & 0.05 [0, 0.20] \\
Frequency of assessment (years)                                  & $f_{\mathrm{assess}}$ & 20 [5, 50] \\
\bottomrule
\end{tabular}
\label{tab:salm-pars}
\end{table}

\clearpage

\section{Figures}

\clearpage

\begin{figure}[htbp]
\centering
\includegraphics[height=5.5in]{../examples/simulation-diagram2.pdf}
\caption{Flow chart of the salmon-metapopulation simulation. There are $n$ 
salmon populations and $t$ generations. Blue text indicates values that are 
generated before the simulation progresses through time. Red text indicates 
steps in which calculations are performed. Black text indicates values that are 
calculated. Grey text indicates parameters that can be set. Green text 
indicates the looping structure of the simulation.}
\label{f:sim-flow}
\end{figure}

\clearpage

\begin{figure}[htbp]
\centering
\includegraphics[width=4.0in]{../examples/spatial-arma-sim.pdf}
\caption{The components of an example metapopulation simulation.  We show, from 
top to bottom, the environmental signal, the resulting productivity parameter 
(Ricker $a$), the salmon returns, fisheries catch, salmon escapement, salmon 
straying from their natal streams, salmon joining from other streams, 
stock-recruit residuals on a log scale, and the estimated $a$ and $b$ 
parameters in the fitted Ricker curve. The colored lines indicate populations 
that thrive at low (cool colours) to high (warm colours) environmental values.}
\label{f:sp-eg}
\end{figure}

\clearpage

\begin{figure}[htbp]
\centering
\includegraphics[width=3.0in]{../examples/thermal-curve-scenarios.pdf}
\caption{Different ways of prioritizing response-diversity conservation. The 
coloured lines represent environmental-tolerance curves for conserved 
populations. The colours identify populations that thrive at low (cool colours) 
to high (warm colours).}
\label{f:curves}
\end{figure}

\clearpage

\begin{figure}[htbp]
\centering
\includegraphics[width=5.0in]{../examples/spatial-mv.pdf}
\caption{The importance of preserving environmental response diversity through 
spatial conservation strategies. The conservation strategies correspond to 
Fig.~\ref{f:curves} and represent conserving a range of responses (green), the 
most stable populations only (orange), or one type of environmental response 
(purple and pink).  In risk-return space we show environmental scenarios that 
are comprised primarily of (a) short-term and (b) long-term environmental 
fluctuations (see Fig.~X). The dots show simulated metapopulations and the 
contours show 25\% and 75\% quantiles across 500 simulations per strategy. We 
also show example metapopulation abundance time series for the (c, e) 
short-term and (d, f) long-term  environmental-fluctuation scenarios.}
\label{f:sp-mv}
\end{figure}

\clearpage

\begin{figure}[htbp]
\centering
\includegraphics[width=5.0in]{../examples/cons-plans-n.pdf}
\caption{The importance of preserving as many salmon populations as possible 
when we don't know how response diversity is distributed. In risk-return space 
we show environmental scenarios that are comprised primarily of (a) short-term 
and (b) long-term environmental fluctuations (see Fig.~X). We show 
metapopulations in which 2 (red), 4 (orange), 8 (yellow), or 16 (green) 
populations of random response diversity are conserved. The dots show simulated 
metapopulations and the contours show 25\% and 75\% quantiles across 500 
simulations per strategy. We also show example metapopulation (c) 
rate-of-change and (d) abundance time series for the short-term 
environmental-fluctuation scenario.}
\label{f:n-mv}
\end{figure}

\clearpage

\begin{figure}[htbp]
\centering
\includegraphics[width=5.0in]{../examples/quasi-extinct.pdf}
\caption{First draft of a quasi-extinction figure. The y-axis shows the 
cumulative proportion of subpopulations that have had abundance dip below a 
threshold of 20\% of unfished equilibrium abundance. We could show a (binomial) 
measure of uncertainty around these if we want to. Some interesting points: the 
cumulative proportion of ``quasi-extinction'' is higher for both ``one half'' 
scenarios. If it's set up correctly, then this means that there's a benefit to 
the subpopulations to investing in a more ``optimal'' portfolio. The }
\label{f:n-mv}
\end{figure}
