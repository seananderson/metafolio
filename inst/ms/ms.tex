\section{Abstract}

{[}@convertino2013; @wade2013{]}

\section{Introduction}

Managing risk is fundamental to the conservation of an endangered species. When an endangered species exists as a metapopulation, we can manage risk at two levels: at the population level or at the metapopulation level {[}REF{]}. Typically we treat sources of risk for metapopulations as exogenous and manage risk by altering fishing or hunting, implementing reserves, or improving connectivity for subpopulations {[}e.g. @akcakaya2007{]}.

The management of financial portfolios provides another way to consider risk. Economists consider the risk and performance of a financial portfolio based on the weighting of individual investments (called assets) that make up the portfolio. Modern Portfolio Theory (MPT) proposes that there is a set of portfolios that maximizes expected return for a level of expected risk or minimizes expected risk for a level of expected return \citep{markowitz1952, markowitz1959}. Similarly, expected growth rate and variance of a metapopulation is a function of the variance, covariance, and size of the individual populations. A portfolio approach to managing risk for a metapopulation might therefore consider how conservation actions affect the weight of each population in a metapopulation portfolio.

Managing Pacific salmon under the uncertainty of climate change is an ideal scenario to consider through the lens of portfolio theory for three reasons. (1) Salmon often form metapopulations {[}REF{]} and we can consider the metapopulation in a river-catchment as a portfolio and the stream populations as assets. Continuing the analogy, predators and fisheries often integrate across multiple streams, acting as investors in the salmon portfolio. Fisheries managers and conservation agencies then act as portfolio managers through actions such choosing which salmon habitat to prioritize for protection or restoration. (2) Many Pacific salmon metapopulations are highly threatened {[}REFs{]} and will likely become more at risk as threats such as overfishing, dams, roads, logging, and particularly climate change intensify {[}REFs throughout{]}. (3) Salmon are highly valued by society, fishers, conservation groups, and indigenous people {[}REFs{]}. Although civil society allocates extensive resources to conserving salmonids (e.g.~XX in XX) the scale of the problem demands a prioritization of conservation efforts {[}REFs{]}.

A diversified financial portfolio reduces risk by exploiting the covariance between assets --- ideally the same market forces cause one stock to rise if another one falls --- and there are two key mechanisms that reduce metapopulation risk by creating asynchronous population dynamics. First, localized habitat features can filter the environment creating unique conditions within a spread out metapopulation {[}REFS{]} (i.e.~Moran effect REF). For example, landscape topology can affect stream flow causing different streams of salmon to experience unique conditions {[}REF{]}. Second, diversity of life-histories, behaviours, personalities, and genetics can cause populations to respond uniquely to the same environment (i.e.~response diversity REF and biocomplexity REF). These unique traits can be phenotypic or genetic and derived from short-term adaptations {[}REF{]} or long-term evolution {[}REF{]} to historical conditions. In this paper we focus on the latter, response-diversity mechanism.

Here, we ask how a portfolio approach to management can inform the conservation of metapopulations in a changing world. We ask two primary questions: (1) What does portfolio theory tell us about spatial approaches to prioritizing metapopulation conservation? (2) If we don't know how response-diversity is distributed, what does portfolio theory tell us about how many populations to conserve? To answer these questions, we develop a salmon metapopulation simulation in which spatially-distributed thermal tolerance and patterns of short- and long-term climatic change drive population-specific productivity. We then implement different conservation ``rules of thumb'' that control the population-level carrying capacities and evaluate the salmon portfolios along risk and return axes, as a financial portfolio manager might. We show that conserving response diversity buffers metapopulation risk given short-term climate forcing and metapopulation growth rate given long-term climate warming. We then show that conserving more subpopulations buffers risk regardless of response diversity or climate trend, and we conclude that considering metapopulations through portfolio theory provides a useful additional dimension to evaluate conservation strategies.

\section{Methods}

We developed a salmon metapopulation simulation model that includes a stock-recruit relationship, demographic stochasticity, straying between populations, varying responses to the environment, escapement target setting, and implementation uncertainty. Under two kinds of environmental regimes we tested different conservation rules of thumb and evaluated these plans in risk-return space similar to how financial managers evaluate financial portfolios. We illustrate the overall simulation structure in Fig.~1. We provide a package \texttt{metafolio} for the statistical software \texttt{R} \citep{r2013} as an appendix, to carry out the simulations and analyses we describe in this paper.

\subsection{Defining the ecological portfolio}

In our ecological portfolios, we defined assets as stream-level populations and the portfolios as salmon metapopulations. We use the terms \emph{stream} and \emph{populations} interchangeably to represent the portfolio assets. We defined the portfolio investors as the stakeholders in the fishery and metapopulation performance. CHANGE THIS For example, we could consider fisheries managers, conservation agencies, or First Nations groups as investors. We defined asset value as the abundance of returning salmon in each stream and value of the portfolio as the overall metapopulation abundance. In this scenario, the equivalent to financial rate of return is the generation-to-generation rate of change of metapopulation abundance. We defined the financial asset investment weights as the capacity of the stream populations --- specifically the unfinished equilibrium stock size --- since maintaining or restoring habitat requires money, time, and resources. Investment in a population therefore represents investing in salmon habitat conservation or reconstruction.

\subsection{Salmon metapopulation dynamics}

The salmon metapopulation dynamics in our simulation were governed by a spawner-return relationship with demographic stochasticity and by straying between populations.

\subsubsection{Spawner-return relationship}

We defined the spawner-return relationship with a Ricker model \citep{ricker1954},

\[R_{i(t)} = S_{i(t)}e^{a_{i(t)}(1-S_{i(t)}/b_i) + w_{i(t)}}\]

\noindent where $i$ represents a population, $t$ a generation time, $R$ the number of returns, $S$ the number of spawners, $a$ the productivity parameter (which can vary with the environment), and $b$ the density-dependent term (which is used as the asset weights in the portfolios). The term $w_{i(t)}$ represents first-order autocorrelated error. Formally, $w_{i(t)} = w_{ti-1} \rho_w + r_{i(t)}$, where $r_{i(t)}$ represents independent and normally distributed error with mean 0 and standard deviation of $\sigma_r$. The parameter $\rho_w$ represents the correlation between residuals from subsequent generations.

We manipulated the capacity and productivity parameters $b$ and $a$ as part of the portfolio simulation. The capacity parameters $b_i$ were controlled by the investment weights in the populations. For example, a large investment in a stream was represented by a larger unfished equilibrium stock size $b$ for stream $i$. The productivity parameters $a_{i(t)}$ were controlled by the interaction between an environmental signal and the stream-level population environmental-tolerance curves.

We generated the environmental-tolerance parabolas according to

\[a_{i(t)} =
  \begin{cases}
    W_i (e_t - e_i^{\mathrm{opt}})^2 + a_i^{\mathrm{max}},
      & \text{if } a_{i(t)} > 0\\
      0, & \text{if } a_{i(t)} \leq 0
  \end{cases}\]

\noindent where $W_i$ controls the width of the curve for population $i$, $e_t$ represents the environmental value at generation $t$, $e_i^{\mathrm{opt}}$ represents the optimal environmental value for population $i$, and $a_i^{\mathrm{max}}$ represents the maximum possible $a$ value for population $i$. We set the $W_i$ parameters (Table S1) and calculated the $a_i^{\mathrm{max}}$ parameters so that the area under each curve $A_i$ was equal. See Figure 2a for example environmental-tolerance curves.

\subsubsection{Straying}

We implemented straying as in \citet{cooper1999}. We set up the metapopulation in a simple scenario: we arranged the populations in a line and those that were nearer to each other were more likely to stray between each other. Two parameters controlled the straying: the fraction of fish $f_{\mathrm{stray}}$ that stray from their natal stream in any generation and the rate $m$ at which this straying between streams decays with distance. We calculated the number of salmon straying from stream $j$ to stream $i$ as

\[\mathrm{strays}_{tij} = f_{\mathrm{stray}} R_{tj}
    \frac{e^{-m \lvert i-j \rvert }}
      {\displaystyle\sum\limits_{
        \substack{k = 1 \\ k \neq j}}^{n} 
        e^{-m \lvert k-j \rvert }}\]

\noindent where $R_{tj}$ is the number of returning salmon at generation $t$ whose natal stream was stream $j$. The subscript $k$ represents a stream ID and $n$ the number of populations. The denominator is a normalizing constant to ensure the desired fraction of fish stray. See Figure \ref{f:stray} for an example straying matrix.

\subsection{Fishing}

Our simulation used a simple set of rules to establish escapement targets and harvest the fish. Every five years our simulation fitted a spawner-return function and target harvest rate $H_{\mathrm{tar}}$ was set based on \citet{hilborn1992} as

\[H_{\mathrm{tar}} = \frac{A}{b (0.5 - 0.07a)}
  \label{eq:esc}\]

\noindent where $A$ represents the return abundance and $a$ and $b$ represent the Ricker model parameters. We included implementation uncertainty in the actual harvest rate $H_{\mathrm{act}}$ as

\[H_{\mathrm{act}} = \mathrm{beta}(\alpha_h, \beta_h)\]

\noindent where $\alpha_h$ and $\beta_h$ are the location and shape parameters in a beta distribution. They can be calculated from the desired mean $H_{\mathrm{tar}}$ and standard deviation $\sigma_h$ as \citep[p.~97]{morgan1990}

\[\begin{aligned}
  \alpha_h &= H_{\mathrm{tar}}^2
                \left(
                \frac{1 - H_{\mathrm{tar}}}{\sigma_h^2} - \frac{1}{H_{\mathrm{tar}}}
                \right)\\
   \beta_h &= \alpha \left({\frac{1}{H_{\mathrm{tar}}} - 1}\right).\end{aligned}\]

Further, to establish a range of spawner-return values and to mimic the start of an open-access fishery, for the first 30 years we drew the fraction of fish harvested randomly from a uniform distribution between 0.1 and 0.9. We discarded these initial 30 years as a burn-in period.

\subsection{Environmental dynamics}

We evaluated portfolio performance under short- and long-term environmental dynamics. We represented short-term dynamics as a stationary first-order autoregressive process, AR(1), with correlation $\rho_e$

\[e_t = e_{t-1} \rho_e + d_t, d_t \sim \mathrm{N}(0, \sigma_d)\]

\noindent where $e_t$ represents the environmental value in generation $t$ and $d$ represents normally distributed deviations of mean 0 and standard deviation $\sigma_d$. We represented long-term environmental dynamics as a linear shift in the environmental value through time

\[e_t = \beta_e t - \overline{\beta_e t}\]

\noindent where $\beta_e$ represents the slope. To maintain a balanced response, we centered the trend by subtracting the mean $\overline{\beta_e t}$ so that midway through the simulation (after any burn-in period) the environmental value was at the mean environmental tolerance.

Mention Fig.~3 and maybe put it before Fig.~2?

\subsection{Conservation rules of thumb}

We evaluated two sets of conservation rules of thumb: (1) spatial response diversity conservation strategies in an idealistic scenario where you can detect response diversity, and (2) a more realistic scenario where we know little about response diversity and we're left with a choice of how many populations to conserve.

We evaluated four spatial conservation rules of thumb (Fig.~2b--e). In all spatial scenarios, we conserved four populations and set the unfished equilibrium biomass of the remaining populations to near elimination (five salmon). These reduced populations could still receive straying salmon but were unlikely to rebuild on their own to a substantial abundance. The four scenarios were:

\begin{enumerate}
\def\labelenumi{\arabic{enumi}.}
\itemsep1pt\parskip0pt\parsep0pt
\item
  Conserve an even sampling of response diversity.
\item
  Conserve the most stable populations only.
\item
  Conserve one half of the metapopulation.
\item
  Conserve the other half of the metapopulation.
\end{enumerate}

In reality we rarely know precise levels of response diversity. We therefore additionally considered a case where the conservation was randomly assigned with respect to response diversity but where different numbers of streams could be conserved. We considered conserving from two to 16 streams. Similarly to the spatial strategies, we reduced the capacity of the remaining streams to the nominal level of five salmon.

\section{Results}

\subsection{Which populations to conserve?}

\subsubsection{Short-term environment}

Given strong short-term environmental fluctuations, conserving response diversity buffers the risk properties of an ecological portfolio (Fig.~4a). In our simulation, the median variance of generation-to-generation rate of change in abundance was X times lower given balanced response diversity (full range of responses or most stable only vs.~conserving one half or the other). In fact, even though by conserving the full range of responses, the portfolio was comprised of warm and cool-thriving populations that were more variable on their own, each was balanced by an opposing population. The portfolio risk was therefore comparable between the full range of responses and most stable only portfolios.

We can see the mechanism behind these portfolio properties by inspecting example population time series (Fig.~4c, d). If only the upper or lower half of response diversity is conserved, the portfolio tends to do well or poorly depending on the environmental conditions (Fig.~4d). This risk is buffered with balanced response diversity (Fig.~4c).

\subsubsection{Long-term environment}

Given long-term environmental change, the choice of which populations to conserve affects the return properties of an ecological portfolio (Fig.~4b). By conserving balanced response diversity, an ecological manager is hedging his or her bets on what will happen with the environment and how the populations will respond. The typical return for a balanced response diversity strategy was zero --- the metapopulation neither increased or decreased in abundance in the long run. By conserving only the upper or lower half of response diversity, a conservation manager is putting all his or her eggs in one basket --- the metapopulation might do really well through time or it might do really poorly. The example metapopulation abundance time series (Fig.~4d, f) illustrate this effect. By conserving response diversity, when one population is doing poorly, another is doing well and the metapopulation abundance remains stationary through time.

Notably, in these simulations, if a managers invested in the populations that were doing well at the beginning they would have had the lowest rate-of-return portfolio in the end (purple portfolios in Fig.~4b).

Spatial conservation strategies in the face of longterm environmental change hinge on whether you ``get it right'' --- whether you choose just the right populations to conserve.

\subsection{How many populations to conserve?}

Given a scenario where we don't know the distribution of population-level response diversity, portfolio optimization informs us about the risk buffering from maintaining multiple populations (Fig.~5). In short, investing in more populations buffers portfolio risk.

\subsubsection{Short-term environment}

Given short-term environmental noise, conserving more populations buffers portfolio risk while the random conservation of response diversity creates a spread of metapopulation risk for the same number of populations conserved (Fig.~5a). For example, a metapopulation with eight conserved populations is X times less risky than a metapopulation with only four. We can see this risk-buffering effect through example metapopulations in Fig.~5c and 5d. We note that the risk-return axes of portfolio optimization ignore the absolute-abundance dimension (Fig.~5d). As one would expect, conserving fewer populations also results in lower-abundance metapopulations.

\subsubsection{Long-term environment}

Given long-term environmental noise, conserving more populations also buffers portfolio risk. However, in contrast to the short-term environmental noise scenario, the unknown response diversity creates a spread of possible metapopulation return for the same number of conserved populations (Fig.~5b). Here, the number of populations conserved buffers non-systematic (i.e.~not-environmentally-driven) stochasticity.

\section{Discussion}

\section{Acknowledgements}

\bibliographystyle{ecologyletters2}

\bibliography{jshort,ms}

\clearpage

\section{Figures}

\clearpage

\begin{figure}[htbp]
\centering
\includegraphics[height=5.5in]{../examples/simulation-diagram2.pdf}
\caption{Flow chart of the salmon-metapopulation simulation. There are $n$ salmon populations and $t$ generations. Blue text indicates values that are generated before the simulation progresses through time. Red text indicates steps in which calculations are performed through time. Black text indicates values that are calculated. Grey text indicates parameters that can be set. Green text indicates the looping structure of the simulation.}
\label{f:sim-flow}
\end{figure}

\clearpage

\begin{figure}[htbp]
\centering
\includegraphics[width=3.0in]{../examples/thermal-curve-scenarios.pdf}
\caption{Different ways of prioritizing response-diversity conservation. Panel a shows the thermal tolerance cures for ten possible populations and panels b--e show different ways of prioritizing four of those populations. The curves describe how productivity varies with the environment for a given population. Some populations thrive at low environmental values (cool colours) and some at high (warm colours) values. Some are tolerant to a wider range of environmental conditions (yellow-to-green colours) but with a lower maximum productivity. The total possible productivity (the area under the curves) is the same for each population.}
\label{f:curves}
\end{figure}

\clearpage

\begin{figure}[htbp]
\centering
\includegraphics[width=4.0in]{../examples/spatial-arma-sim.pdf}
\caption{The components of an example metapopulation simulation.  We show, from top to bottom, the environmental signal, the resulting productivity parameter (Ricker $a$), the salmon returns, fisheries catch, salmon escapement, salmon straying from their natal streams, salmon joining from other streams, stock-recruit residuals on a log scale, and the estimated $a$ and $b$ parameters in the fitted Ricker curve. The colored lines indicate populations that thrive at low (cool colours) to high (warm colours) environmental values.}
\label{f:sp-eg}
\end{figure}

\clearpage

\begin{figure}[htbp]
\centering
\includegraphics[width=5.0in]{../examples/spatial-mv.pdf}
\caption{The importance of preserving environmental response diversity through spatial conservation strategies. The conservation strategies correspond to Fig.~\ref{f:curves} and represent conserving a range of responses (green), the most stable populations only (orange), or one type of environmental response (purple and pink).  In risk-return space we show environmental scenarios that are comprised primarily of (a) short-term and (b) long-term environmental fluctuations (see Fig.~X). The dots show simulated metapopulations and the contours show 25\% and 75\% quantiles across 500 simulations per strategy. We also show example metapopulation abundance time series for the (c, e) short-term and (d, f) long-term  environmental-fluctuation scenarios.}
\label{f:sp-mv}
\end{figure}

\clearpage

\begin{figure}[htbp]
\centering
\includegraphics[width=5.0in]{../examples/cons-plans-n.pdf}
\caption{The importance of preserving as many subpopulations as possible when we don't know how response diversity is distributed. In risk-return space we show environmental scenarios that are comprised primarily of (a) short-term and (b) long-term environmental fluctuations (see Fig.~X). We show metapopulations in which 2 (red), 4 (orange), 8 (yellow), or 16 (green) populations of random response diversity are conserved. The dots show simulated metapopulations and the contours show 25\% and 75\% quantiles across 500 simulations per strategy. We also show example metapopulation (c) rate-of-change and (d) abundance time series for the short-term environmental-fluctuation scenario.}
\label{f:n-mv}
\end{figure}

\clearpage
