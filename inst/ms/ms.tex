\emph{Other title ideas:}

\begin{itemize}
\item
  Portfolio optimization reveals\ldots{}
\item
  something more general with animals instead of just salmon?
\item
  Portfolio theory reveals conservation strategies that\ldots{}
\item
  Intelligent tinkering with ecological portfolios: patterns of
  environmental response diversity drive salmon conservation priorities
\item
  Keeping every cog and wheel\ldots{}
\end{itemize}

\section{Abstract}

\begin{enumerate}
\def\labelenumi{\arabic{enumi}.}
\item
  Managing risk is fundamental to the conservation of an endangered
  species. When an endangered species exists as a metapopulation, we can
  manage risk at two levels: at the population level or at the
  metapopulation level. Whereas risk is typically managed at the
  population level, a portfolio approach to managing risk might consider
  how conservation affects the ``weight'' of each population in a
  metapopulation ``portfolio''.
\item
  Here, we ask how a portfolio approach to managing risk can inform the
  spatial conservation of metapopulations in a changing world. To answer
  this, we develop a salmon metapopulation simulation in which
  population-specific productivity is driven by spatially-distributed
  environmental tolerance and patterns of short- and long-term
  environmental change. We then implement different spatial conservation
  ``rules of thumb'' that control the population-specific carrying
  capacities and evaluate the salmon portfolios along risk and return
  axes, similarly to how financial portfolios are assessed.
\item
  Our results show, first, that maintaining populations with a variety
  of environmental tolerances gives the best chance at an efficient
  ecological portfolio --- minimizing metapopulation variance while
  maximizing metapopulation growth rate. This finding emphasizes the
  risk of allowing large spatial blocks of habitat destruction, say
  through the development of dams. Second, we show that focusing on
  well-performing stocks now at the detriment of others is at best
  equivalent to a risky but efficient portfolio and is more likely a
  risky and inefficient portfolio --- it neither minimizes
  metapopulation variance nor maximizes growth rate compared to other
  strategies. Third, we show that maintaining more populations reduces
  metapopulation risk for the same spatial conservation strategy. Given
  a lack of knowledge of how populations respond to the environment, the
  most risk-averse approach is to conserve as many populations as
  possible.
\item
  Our findings highlight three key points: (1) the conservation priority
  of maintaining biocomplexity and therefore environmental response
  diversity, (2) the research priority of identifying differences in
  environmental tolerance given predicted environmental changes, and (3)
  the utility of considering risk for groups of fish stocks ---
  especially given environmental, biological, and implementation
  uncertainty --- through the lens of portfolio theory.
\end{enumerate}

\section{Introduction}

Managing risk is fundamental to the conservation of an endangered
species. When an endangered species exists as a metapopulation, we can
manage risk at two levels: at the population level or at the
metapopulation level. Typically we treat sources of risk at the
metapopulation level as external and uncontrollable and so we manage
risk by altering fishing or hunting on a population level as well as
improving the connectivity of populations.

The management of financial portfolios provides another way of
considering risk. Economists consider the risk and performance of a
financial portfolio as a function of the weighting of individual assets
that make up the portfolio. Modern Portfolio Theory (MPT) proposes that
there is a set of portfolios that maximizes expected return for a level
of expected risk or minimizes expected risk for a level of expected
return \citep{markowitz1952, markowitz1959}. This optimal set contains
portfolios that range along a continuum of risk-tolerance; economists
refer to this set as the efficient frontier.

Portfolios provide another way of managing risk - explain mean-variance
optimization and how this translates to ecology
\citep{figge2004}\citep{hoekstra2012}
\citep[\citet{ando2012}]{ando2011}\citep[\citet{markowitz1959}]{markowitz1952}

We use salmon as an example, describe the components of the example,
provide references on metapopulation existence \citep{schindler2010}

Research questions, and briefly our approach - Monte Carlo simulation
modelling, as simple a model as possible while retaining important
realistic aspects of salmon metapopulations

Here, we ask how a portfolio approach to managing risk can inform the
spatial conservation of metapopulations in a changing world. To answer
this, we develop a salmon metapopulation simulation in which
population-specific productivity is driven by spatially-distributed
environmental tolerance and patterns of short- and long-term
environmental change. We then implement different conservation ``rules
of thumb'' that control the population-specific carrying capacities and
evaluate the salmon portfolios along risk and return axes, similarly to
how financial portfolios are assessed. We show\ldots{} (briefly)

\section{Methods}

We developed a salmon metapopulation simulation model that includes a
stock-recruit relationship, demographic stochasticity, straying between
populations, varying responses to the environment, escapement target
setting, and implementation uncertainty. Under two kinds of
environmental regimes we tested different conservation rules of thumb
and evaluated these plans in risk-return space similar to how financial
managers evaluate financial portfolios. See Figure \ref{f:sim-flow} for
an illustration of the overall simulation structure. We provide a
package \texttt{metafolio} for the statistical software \texttt{R}
\citep{r2013} as an appendix, which can be used to carry out the
simulations and analyses we describe in this paper.

\subsection{Defining the ecological portfolio}

In our ecological portfolios, we defined assets as stream-level
populations and the portfolios as salmon metapopulations (Table
\ref{t:port}). We use the terms \textit{stream} and \textit{populations}
interchangeably to represent the portfolio assets. The portfolio
investors were the stakeholders in the fishery and metapopulation
performance. For example, we could consider fisheries managers,
conservation agencies, or First Nations groups as investors. We defined
asset value as the abundance of returning salmon in each stream and
value of the portfolio as the overall metapopulation abundance. In this
scenario, the equivalent to financial rate of return is the
generation-to-generation rate of change of metapopulation abundance. We
defined the financial asset investment weights as the carrying capacity
of the stream populations --- specifically the unfinished equilibrium
stock size --- since maintaining or restoring habitat requires money,
time, and resources.

\subsection{Salmon metapopulation dynamics}

The salmon metapopulation dynamics in our simulation were governed by a
spawner-return relationship with stochastic noise and by straying
between populations.

\subsubsection{Spawner-return relationship}

We defined the spawner-return relationship with a Ricker model
\citep{ricker1954},

\begin{equation}
R_{ti} = S_{ti}e^{a_{ti}(1-S_{ti}/b_i) + w_{ti}}
\end{equation}

\noindent where $t$ represents a generation time, $i$ represents a
population, $R$ is the number of returns, $S$ is the number of spawners,
$a$ is the productivity parameter (which can vary with the environment),
and $b$ is the density-dependent term (which is used as the asset
weights in the portfolios). The term $w_{ti}$ represents first-order
autocorrelated error. Formally, $w_{ti} = w_{ti-1} \rho_w + v_{ti}$,
where $v_{ti}$ represents independent and normally distributed error
with mean 0 and standard deviation of $\sigma_v$. The parameter $\rho_w$
represents the correlation between residuals from subsequent
generations.

We manipulated the capacity and productivity parameters $b$ and $a$ as
part of the portfolio simulation. The capacity parameters $b_i$ were
controlled by the investment weights in the populations. For example, a
large investment in a stream was represented by a larger unfished
equilibrium stock size $b$ for stream $i$. The productivity parameters
$a_{ti}$ were controlled by the interaction between an environmental
signal and the stream-level population environmental tolerance curves.
The environmental tolerance curves were generated as parabolas according
to

\begin{equation}
  a_{ti} =
  \begin{cases}
    W_i (e_t - e_i^{\mathrm{opt}})^2 + a_i^{\mathrm{max}},
      & \text{if } a_{ti} > 0\\
      0, & \text{if } a_{ti} \leq 0
  \end{cases}
\end{equation}

\noindent where $W_i$ controls the width of the curve for population
$i$, $e_t$ represents the environmental value at generation $t$,
$e_i^{\mathrm{opt}}$ represents the optimal environmental value for
population $i$, and $a_i^{\mathrm{max}}$ represents the maximum possible
$a$ value for population $i$. See Figure \ref{f:all-curves} for example
environmental tolerance curves.

\subsubsection{Straying}

We implemented straying as in \citet{cooper1999}. We set up the
metapopulation in a simple scenario: we arranged the populations in a
line and those that were nearer to each other were more likely to stray
between each other. Two parameters controlled the straying: the fraction
of fish $f_{\mathrm{stray}}$ that stray from their natal stream in any
generation and the rate $m$ at which this straying between streams
decays with distance. We calculated the number of salmon straying from
stream $j$ to stream $i$ as

\begin{equation}
  \mathrm{strays}_{tij} = f_{\mathrm{stray}} R_{tj}
    \frac{e^{-m \lvert i-j \rvert }}
      {\displaystyle\sum\limits_{\substack{k = 1 \\
    k \neq j}}^{n} e^{-m \lvert k-j \rvert }}
  \label{eq:stray}
\end{equation}

\noindent where $R_{tj}$ is the number of returning salmon at generation
$t$ whose natal stream was stream $j$. The subscript $k$ represents a
stream ID and $n$ the number of populations. The denominator is a
normalizing constant to ensure the desired fraction of fish stray. See
Figure \ref{f:stray} for an example straying matrix.

\subsection{Fishing}

Our simulation used a simple set of rules to establish escapement
targets and harvest the fish. Every $f_\mathrm{assess}$ years (base case
of five years) our simulation fitted a spawner-return function and
target harvest rate $H_{\mathrm{tar}}$ was set based on
\citet{hilborn1992} as

\begin{equation}
  H_{\mathrm{tar}} = \frac{A}{b (0.5 - 0.07a)}
  \label{eq:esc}
\end{equation}

\noindent where $A$ represents the return abundance and $a$ and $b$
represent the Ricker model parameters. We included implementation
uncertainty in the actual harvest rate $H_{\mathrm{act}}$ as

\begin{equation}
  H_{\mathrm{act}} = \mathrm{beta}(\alpha_h, \beta_h)
\end{equation}

\noindent where $\alpha_h$ and $\beta_h$ are the location and shape
parameters in a beta distribution. They can be calculated from the
desired mean $\mu_h$ and standard deviation $\sigma_h$ as
\citep[p.~97]{morgan1990}

\begin{align}
  \alpha_h &= \mu_h^2
                \left(
                \frac{1 - \mu_h}{\sigma_h^2} - \frac{1}{\mu_h}
                \right)\\
   \beta_h &= \alpha \left({\frac{1}{\mu_h} - 1}\right).
\end{align}

\noindent Further, to establish a range of spawner-return values and to
mimic the start of an open-access fishery, for the first 20 (TODO
variable) years we drew the fraction of fish harvested randomly from a
uniform distribution between 0.1 and 0.9. TODO CONSIDER BETTER RANGE
HERE

\subsection{Environmental dynamics}

We evaluated portfolio performance under short- and long-term
environmental dynamics. We represented short-term dynamics as a
stationary first-order autoregressive process, AR(1), with correlation
$\rho_e$

\begin{equation}
  e_t = e_{t-1} \rho_e + d_t, d_t \sim \mathrm{N}(0, \sigma_d)
\end{equation}

\noindent where $e_t$ represents the environmental value in generation
$t$ and $d$ represents normally distributed deviations of mean 0 and
standard deviation $\sigma_d$. We represented long-term environmental
dynamics as a linear shift in the environmental value through time

\begin{equation}
  e_t = \alpha_e + \beta_e t
\end{equation}

\noindent where $\alpha_e$ represents the intercept and $\beta_e$ the
slope. To maintain a balanced response, we centered the trend so that
midway through the simulation (after any burn-in period) the
environmental value was at the mean environmental tolerance.

\subsection{Conservation rules of thumb}

We evaluated two sets of conservation rules of thumb: (1) spatial
response diversity rules of thumb in an idealistic scenario where you
can detect response diversity, and (2) a more realistic scenario where
we know little about response diversity and we're left with a choice of
how many populations to conserve. We then considered how optimal these
strategies were across all possible conservation-prioritization
strategies.

\subsubsection{Spatial conservation strategies}

We evaluated four spatial conservation rules of thumb (Figure
\ref{f:curves}). In all spatial scenarios, we conserved four populations
and set the unfished equilibrium biomass of the remaining populations to
near elimination (five salmon). These reduced populations could
therefore still receive straying salmon but were unlikely to re-build on
their own to a substantial abundance. The four scenarios were:

\begin{enumerate}
\def\labelenumi{\arabic{enumi}.}
\itemsep1pt\parskip0pt\parsep0pt
\item
  Conserve an even sampling of response diversity. In other words,
  conserve alternating subpopulations.
\item
  Conserve the most stable populations only.
\item
  Conserve one half of the metapopulation.
\item
  Conserve the other half of the metapopulation.
\end{enumerate}

\subsubsection{How many populations to conserve?}

In reality we rarely know precise levels of response diversity. We
therefore additionally considered a case where the conservation was
randomly assigned with respect to response diversity but where different
numbers of streams can be conserved. We considered conserving from two
to 16 streams. Similarly to the spatial strategies, we reduced the
carrying capacity of the remaining streams to the nominal level of five
salmon.

\subsubsection{How optimal are the conservation strategies?}

To test how optimal the strategies were, we generated metapopulation
portfolios in which the investment weights (Ricker $b$ parameters) were
randomly assigned such that the total available ``investment'' remained
constant. This Monte-Carlo strategy illustrates a large sample of
possible portfolios in risk-return space. We then placed the
conservation strategies listed above into this space to determine how
close to the efficient frontier they lie. {[}Figure not shown yet{]}

\section{Results}

Figure \ref{f:curves} Figure \ref{f:sp-mv} Figure \ref{f:n-mv}

Figure \ref{f:eg-n-linear} Figure \ref{f:eg-n-arma} Figure
\ref{f:eg-sp-arma} Figure \ref{f:eg-sp-linear}

\section{Discussion}

\section{Acknowledgements}

\bibliographystyle{apalike}

\bibliography{jshort,ms}

\clearpage

\section{Tables}

\begin{table}[h!]
\centering
\small
\caption{Components of salmon metapopulation portfolios}
\begin{tabular}{p{3.6cm}p{7.5cm}}
\toprule
Component          & Definition for the salmon portfolio\\
\midrule
Assets             & Stream-level salmon populations; possibly a Viable Salmonid Population\\
Portfolio          & The salmon metapopulation; possibly an Evolutionarily Significant Unit\\
Portfolio managers & Salmon managers\\
Investors          & Salmon managers, conservation agency, or salmon fishers\\
Asset weights      & Carrying capacity (specifically the $b$ parameters in a Ricker model)\\
Asset returns      & Rate of change of generation-to-generation salmon metapopulation abundance\\
Asset risk         & Variance of generation-to-generation salmon metapopulation abundance\\
\bottomrule
\end{tabular}
\label{t:port}
\end{table}

\clearpage

\begin{table}[h!]
\centering
\small
\caption{Salmon metapopulation parameters with base case and alternate values.}
\begin{tabular}{p{7.0cm}p{1.6cm}p{3.2cm}}
\toprule
Description                                                      & Parameter             & Base [lower, upper] \\
\midrule
Stock-recruit residual standard deviation (on log scale)         & $\sigma_v$            & 0.30 [0.05, 0.50] \\
First order AR(1) serial correlation of stock-recruit residuals  & $\rho_w$              & 0.40 [0, 0.80] \\
Fraction of fish that stray from natal streams                   & $f_{\mathrm{stray}}$  & 0.02 [0, 0.10] \\
Exponential rate of decay of straying with distance              & $m$                   & 0.3 [0.05, 0.5] \\
Standard deviation of beta distribution for implementation error & $\sigma_{h}$          & 0.05 [0, 0.20] \\
Frequency of assessment (years)                                  & $f_{\mathrm{assess}}$ & 5 [1, 20] \\
Environmental tolerance parameters ... & & \\
\bottomrule
\end{tabular}
\label{t:pars}
\end{table}

\clearpage

\section{Figures}

\clearpage

\begin{figure}[htbp]
\centering
\includegraphics[height=5.5in]{../examples/simulation-diagram2.pdf}
\caption{Flow chart of the salmon-metapopulation simulation. There are $n$ salmon populations and $t$ generations. Blue text indicates values that are generated before the simulation progresses through time. Red text indicates steps in which calculations are performed through time. Black text indicates values that are calculated. Grey text indicates parameters that can be set. Green text indicates the looping structure of the simulation.}
\label{f:sim-flow}
\end{figure}

\clearpage

\begin{figure}[htbp]
\centering
\includegraphics[width=4.0in]{../examples/spatial-arma-sim.pdf}
\caption{The components of an example metapopulation simulation.  We show, from top to bottom, the environmental signal, the resulting productivity parameter (Ricker $a$), the salmon returns, fisheries catch, salmon escapement, salmon straying from their natal streams, salmon joining from other streams, stock-recruit residuals on a log scale, and the estimated $a$ and $b$ parameters in the fitted Ricker curve. The colored lines indicate populations that thrive at low (cool colours) to high (warm colours) environmental values.}
\label{f:sp-eg}
\end{figure}

\clearpage

\begin{figure}[htbp]
\centering
\includegraphics[width=3.0in]{../examples/thermal-curve-scenarios.pdf}
\caption{Different ways of prioritizing response-diversity conservation. Panel (a) shows the thermal tolerance cures for ten possible populations and panels (b--e) show different ways of prioritizing four of those populations. The curves describe how productivity varies with the environment for a given population. Some populations thrive at low environmental values (cool colours) and some at high (warm colours) values. Some are tolerant to a wider range of environmental conditions (yellow-to-green colours) but with a lower maximum productivity. The total possible productivity across environmental values (the area under the curves) is the same.}
\label{f:curves}
\end{figure}

\clearpage

\begin{figure}[htbp]
\centering
\includegraphics[width=5.0in]{../examples/spatial-mv.pdf}
\caption{The importance of preserving environmental response diversity through spatial conservation strategies. The conservation strategies correspond to Fig.~\ref{f:curves} and represent conserving a range of responses (green), the most stable populations only (orange), or one type of environmental response (purple and pink).  In risk-return space we show environmental scenarios that are comprised primarily of (a) short-term and (b) long-term environmental fluctuations (see Fig.~X). The dots show simulated metapopulations and the contours show 25\% and 75\% quantiles across 500 simulations per strategy. We also show example metapopulation abundance time series for the (c, e) short-term and (d, f) long-term  environmental-fluctuation scenarios.}
\label{f:sp-mv}
\end{figure}

\clearpage

\begin{figure}[htbp]
\centering
\includegraphics[width=5.0in]{../examples/cons-plans-n.pdf}
\caption{The importance of preserving as many salmon populations as possible when we don't know how response diversity is distributed. In risk-return space we show environmental scenarios that are comprised primarily of (a) short-term and (b) long-term environmental fluctuations (see Fig.~X). We show metapopulations in which 2 (red), 4 (orange), 8 (yellow), or 16 (green) populations of random response diversity are conserved. The dots show simulated metapopulations and the contours show 25\% and 75\% quantiles across 500 simulations per strategy. We also show example metapopulation (c) rate-of-change and (d) abundance time series for the short-term environmental-fluctuation scenario.}
\label{f:n-mv}
\end{figure}

\clearpage

\begin{figure}[htbp]
\centering
\includegraphics[width=5.0in]{../examples/quasi-extinct.pdf}
\caption{First draft of a quasi-extinction figure I THINK THIS WILL BE 
SCRAPPED OR CHANGED. The y-axis shows the cumulative proportion of 
subpopulations that have had abundance dip below a threshold of 20\% 
of unfished equilibrium abundance. We could show a (binomial) 
measure of uncertainty around these if we want to. Some interesting points: the 
cumulative proportion of ``quasi-extinction'' is higher for both ``one half'' 
scenarios. If it's set up correctly, then this means that there's a benefit to 
the subpopulations to investing in a more ``optimal'' portfolio. The }
\label{f:ext}
\end{figure}
