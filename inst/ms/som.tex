\section{Supplementary Information}

\subsection{Supplementary code}

The \texttt{metafolio} \texttt{R} package and documentation. Some
details on what you can do with the package.

\subsection{Supplementary figures}

\begin{figure}[htbp]
\centering
\includegraphics[width=4.0in]{../examples/figure/plot-various-options-ts-3pops.pdf}
\caption{The impact of increasing or decreasing various parameter values in the simulations. The different lines represent different salmon populations. (NEED TO ADD PARAMETER VALUES AND EXPAND THIS SLIGHTLY)}
\label{f:eg-sens}
\end{figure}

\clearpage

\begin{figure}[htbp]
\centering
\includegraphics[width=4.0in]{../examples/figure/stray-matrix.pdf}
\caption{An example straying matrix. The rows and columns represent different 
populations (indicated by population number). Dark blue indicates a high rate 
of straying and light blue indicates a low rate of straying.}
\label{f:stray}
\end{figure}

\clearpage

\begin{figure}[htbp]
\centering
\includegraphics[width=4.5in]{../examples/spatial-arma-sim.pdf}
\caption{Spatial and short-term environmental fluctuations}
\label{f:eg-sp-arma}
\end{figure}

\clearpage

\begin{figure}[htbp]
\centering
\includegraphics[width=4.5in]{../examples/spatial-linear-sim.pdf}
\caption{Spatial and long-term environmental fluctuations}
\label{f:eg-sp-linear}
\end{figure}

\clearpage

\begin{figure}[htbp]
\centering
\includegraphics[width=4.5in]{../examples/n-arma-sim.pdf}
\caption{Number and short-term environmental fluctuations}
\label{f:eg-n-arma}
\end{figure}

\clearpage

\begin{figure}[htbp]
\centering
\includegraphics[width=4.5in]{../examples/n-linear-sim.pdf}
\caption{Number and long-term environmental change}
\label{f:eg-n-linear}
\end{figure}

\clearpage
