\documentclass[11pt]{article}
\usepackage{geometry}
\geometry{letterpaper}
\usepackage{amssymb}
\usepackage{amsmath}
\usepackage{amsfonts}
\usepackage{graphicx}
\usepackage{setspace}
%\usepackage[round]{natbib}
%\bibpunct{(}{)}{;}{a}{}{;}
\textheight 20.0cm
\usepackage{lineno}
\usepackage{xcolor}
\renewcommand\linenumberfont{\normalfont\tiny\sffamily\color{gray}}
\usepackage{booktabs}
\usepackage{cite}

\newlabel{t:pars}{{S1}{2}}
\newlabel{f:flowchart}{{S1}{3}}
\newlabel{f:stray}{{S2}{4}}
\newlabel{f:eg-sim-ts}{{S3}{5}}
\newlabel{f:eg-sens}{{S4}{6}}
\newlabel{f:eg-sp-arma-full}{{S5}{7}}
\newlabel{f:eg-sp-arma-half}{{S6}{8}}
\newlabel{f:ret-corr}{{S7}{9}}
\newlabel{f:eg-sp-linear-full}{{S8}{10}}
\newlabel{f:eg-sp-linear-half}{{S9}{11}}
\newlabel{f:eg-n-arma-two}{{S10}{12}}
\newlabel{f:eg-n-arma-sixteen}{{S11}{13}}
\newlabel{f:eg-n-linear-two}{{S12}{14}}
\newlabel{f:eg-n-linear-sixteen}{{S13}{15}}

\DeclareGraphicsExtensions{.pdf}

\widowpenalty=10000
\clubpenalty=10000

\usepackage{titlesec}
\titlespacing\section{0pt}{6pt plus 4pt minus 2pt}{-8pt plus 2pt minus 2pt}
\titlespacing\subsection{0pt}{6pt plus 4pt minus 2pt}{-8pt plus 2pt minus 2pt}

% remove numbers in front of sections:
\makeatletter
\renewcommand\@seccntformat[1]{}
\makeatother

\hyphenation{meta-pop-ulation meta-pop-ulations sub-pop-ulations sub-pop-ulation e-con-o-mist en-vi-ron-men-tal pri-or-i-ti-za-tion}

\begin{document}
%\raggedright

\linenumbers
\modulolinenumbers[2]
\begin{spacing}{1.6}
\setlength{\parindent}{0cm}
\emph{Other title ideas:}

\begin{itemize}
\item
  Prioritizing metapopulation conservation through the lens of portfolio theory
\item
  Portfolio optimization reveals\ldots{}
\item
  Portfolio theory reveals conservation strategies that\ldots{}
\item
  Intelligent tinkering with ecological portfolios: patterns of environmental response diversity drive salmon conservation priorities
\item
  Keeping every cog and wheel\ldots{}
\end{itemize}

\textbf{Ecology Letters}

\begin{itemize}
\itemsep1pt\parskip0pt\parsep0pt
\item
  letters: 5000 words, 6 figures or tables
\item
  ideas and perspectives: 7500 words, 10 figures (needs 300 word proposal)
\end{itemize}

\begin{quote}
Ecology Letters is a forum for the very rapid publication of the most novel research in ecology, research that is not yet in the public domain. Manuscripts relating to the ecology of all taxa, in any biome and geographic area will be considered, and priority will be given to those papers exploring or testing clearly stated hypotheses. The journal publishes concise papers that merit urgent publication by virtue of their originality, general interest and their contribution to new developments in ecology. We discourage purely descriptive papers and those merely confirming or extending results of previous work.
\end{quote}

\textbf{Ecological Applications}

\begin{itemize}
\itemsep1pt\parskip0pt\parsep0pt
\item
  60 pages including everything
\end{itemize}

\begin{quote}
Ecological Applications is concerned broadly with the applications of ecological science to environmental problems. It publishes papers that develop scientific principles to support environmental decision-making, as well as papers that discuss the application of ecological concepts to environmental issues, policy, and management. Papers may report on experimental tests, actual applications, scientific decision support techniques, economic analyses, social implications of environmental issues, or other relevant topics. Statistical or experimental methods papers that support research and applications are welcome. Papers submitted to Ecological Applications should be accessible to both scholars and practitioners.
\end{quote}

\section{Long and rambling abstract that's a bit old now}

\begin{enumerate}
\def\labelenumi{\arabic{enumi}.}
\item
  Managing risk is fundamental to the conservation of an endangered species. When an endangered species exists as a metapopulation, we can manage risk at two levels: at the population level or at the metapopulation level. Whereas risk is typically managed at the population level, a portfolio approach to managing risk might consider how conservation affects the ``weight'' of each population in a metapopulation ``portfolio''.
\item
  Here, we ask how a portfolio approach to managing risk can inform the spatial conservation of metapopulations in a changing world. To answer this, we develop a salmon metapopulation simulation in which population-specific productivity is driven by spatially-distributed environmental tolerance and patterns of short- and long-term environmental change. We then implement different spatial conservation ``rules of thumb'' that control the population-specific carrying capacities and evaluate the salmon portfolios along risk and return axes, similarly to how financial portfolios are assessed.
\item
  Our results show, first, that maintaining populations with a variety of environmental tolerances gives the best chance at an efficient ecological portfolio --- minimizing metapopulation variance while maximizing metapopulation growth rate. This finding emphasizes the risk of allowing large spatial blocks of habitat destruction, say through the development of dams. Second, we show that focusing on well-performing stocks now at the detriment of others is at best equivalent to a risky but efficient portfolio and is more likely a risky and inefficient portfolio --- it neither minimizes metapopulation variance nor maximizes growth rate compared to other strategies. Third, we show that maintaining more populations reduces metapopulation risk for the same spatial conservation strategy. Given a lack of knowledge of how populations respond to the environment, the most risk-averse approach is to conserve as many populations as possible.
\item
  Our findings highlight three key points: (1) the conservation priority of maintaining biocomplexity and therefore environmental response diversity, (2) the research priority of identifying differences in environmental tolerance given predicted environmental changes, and (3) the utility of considering risk for groups of fish stocks --- especially given environmental, biological, and implementation uncertainty --- through the lens of portfolio theory.
\end{enumerate}

\section{Introduction}

\begin{itemize}
\itemsep1pt\parskip0pt\parsep0pt
\item
  Managing risk is fundamental to the conservation of an endangered species.

  \begin{itemize}
  \itemsep1pt\parskip0pt\parsep0pt
  \item
    risk is usually assessed on a population-by-population basis
  \item
    for metapopulations, we can consider risk for components or risk for the whole
  \end{itemize}
\item
  The management of financial portfolios provides another way of considering risk for metapopulations.

  \begin{itemize}
  \itemsep1pt\parskip0pt\parsep0pt
  \item
    performance of weighted assets
  \item
    efficient frontier concept
  \item
    the parallels with ecological metapopulations
  \end{itemize}
\item
  Pacific salmon are an ideal model to consider under the framework of portfolio prioritization because of their metapopulation structure, their societal importance and threatened status, and the need to prioritize conservation efforts.

  \begin{itemize}
  \itemsep1pt\parskip0pt\parsep0pt
  \item
    portfolio-like structure

    \begin{itemize}
    \itemsep1pt\parskip0pt\parsep0pt
    \item
      documented metapopulation structure
    \item
      predators and fisheries often integrate across stocks
    \end{itemize}
  \item
    Highly threatened; highly valued; face many threats

    \begin{itemize}
    \itemsep1pt\parskip0pt\parsep0pt
    \item
      damns
    \item
      roads
    \item
      other habitat degradation
    \item
      logging
    \item
      fishing
    \item
      changing climate
    \end{itemize}
  \item
    There's a need for prioritization

    \begin{itemize}
    \itemsep1pt\parskip0pt\parsep0pt
    \item
      because of the highly threatened nature
    \item
      conflicting goals for resource use
    \item
      limited resources
    \item
      Existing prioritization schemes
    \end{itemize}
  \end{itemize}
\item
  Portfolio stabilization through systematic asynchrony in salmon population dynamics can be driven by stocks responding uniquely to the same environment (`response diversity') or by physical features of the environment causing stocks to experience different environmental forces (`environmental filter').

  \begin{itemize}
  \itemsep1pt\parskip0pt\parsep0pt
  \item
    different responses to the environment: response diversity / biocomplexity

    \begin{itemize}
    \itemsep1pt\parskip0pt\parsep0pt
    \item
      genetic variation by stock
    \item
      can be driven through historical conditions
    \item
      we focus on this here
    \end{itemize}
  \item
    experiencing different environment: environmental filter concept (Moran like)

    \begin{itemize}
    \itemsep1pt\parskip0pt\parsep0pt
    \item
      spatially spread out, geographical and hydrological features filter the environment to create a variety of forces
    \item
      Typically a combination of both
    \end{itemize}
  \item
    Conversely, forces acting against asynchrony

    \begin{itemize}
    \itemsep1pt\parskip0pt\parsep0pt
    \item
      inverse: loss of population diversity
    \item
      inverse: moran
    \item
      but also: dispersal
    \item
      and: trophic interactions with synchronized species
    \end{itemize}
  \end{itemize}
\end{itemize}

Here, we ask how a portfolio approach to management can inform the conservation of metapopulations in a changing world. We ask two primary questions: (1) What does portfolio optimization tell us about different spatial approaches to prioritizing metapopulation conservation assuming that response diversity is spatially distributed? (2) If we don't know how response diversity is distribute, what does portfolio optimization tell us about how many populations we should conserve? To answer these questions, we develop a salmon metapopulation simulation in which population-specific productivity is driven by spatially-distributed environmental tolerance and patterns of short- and long-term climatic change. We then implement different conservation ``rules of thumb'' that control the population-level carrying capacities and evaluate the salmon portfolios along risk and return axes, as a portfolio manager would. Our findings illustrate that: (1) conserving response diversity buffers metapopulation risk given short-term climate forcing and rate of return given long-term climate forcing, (2) conserving more subpopulations buffers risk regardless of the climate trend, and (3) considering metapopulations through the lens of portfolio theory provides a useful additional dimension through which we can evaluate conservation strategies.

\section{Methods}

We developed a salmon metapopulation simulation model that includes a stock-recruit relationship, demographic stochasticity, straying between populations, varying responses to the environment, escapement target setting, and implementation uncertainty. Under two kinds of environmental regimes we tested different conservation rules of thumb and evaluated these plans in risk-return space similar to how financial managers evaluate financial portfolios. See Figure \ref{f:sim-flow} for an illustration of the overall simulation structure. See Table \ref{t:pars} for a list of the input parameters to our simulation and their default values. We provide a package \texttt{metafolio} for the statistical software \texttt{R} \citep{r2013} as an appendix, to carry out the simulations and analyses we describe in this paper.

\subsection{Defining the ecological portfolio}

In our ecological portfolios, we defined assets as stream-level populations and the portfolios as salmon metapopulations (Table \ref{t:port}). We use the terms \emph{stream} and \emph{populations} interchangeably to represent the portfolio assets. We defined the portfolio investors as the stakeholders in the fishery and metapopulation performance. For example, we could consider fisheries managers, conservation agencies, or First Nations groups as investors. We defined asset value as the abundance of returning salmon in each stream and value of the portfolio as the overall metapopulation abundance. In this scenario, the equivalent to financial rate of return is the generation-to-generation rate of change of metapopulation abundance. We defined the financial asset investment weights as the capacity of the stream populations --- specifically the unfinished equilibrium stock size --- since maintaining or restoring habitat requires money, time, and resources. Investment in a population therefore represents investing in salmon habitat conservation or reconstruction.

\subsection{Salmon metapopulation dynamics}

The salmon metapopulation dynamics in our simulation were governed by a spawner-return relationship with demographic stochasticity and by straying between populations.

\subsubsection{Spawner-return relationship}

We defined the spawner-return relationship with a Ricker model \citep{ricker1954},

\begin{equation}
R_{ti} = S_{ti}e^{a_{ti}(1-S_{ti}/b_i) + w_{ti}}
\end{equation}

\noindent where $t$ represents a generation time, $i$ represents a population, $R$ is the number of returns, $S$ is the number of spawners, $a$ is the productivity parameter (which can vary with the environment), and $b$ is the density-dependent term (which is used as the asset weights in the portfolios). The term $w_{ti}$ represents first-order autocorrelated error. Formally, $w_{ti} = w_{ti-1} \rho_w + r_{ti}$, where $r_{ti}$ represents independent and normally distributed error with mean 0 and standard deviation of $\sigma_r$. The parameter $\rho_w$ represents the correlation between residuals from subsequent generations.

We manipulated the capacity and productivity parameters $b$ and $a$ as part of the portfolio simulation. The capacity parameters $b_i$ were controlled by the investment weights in the populations. For example, a large investment in a stream was represented by a larger unfished equilibrium stock size $b$ for stream $i$. The productivity parameters $a_{ti}$ were controlled by the interaction between an environmental signal and the stream-level population environmental-tolerance curves.

We generated the environmental-tolerance parabolas according to

\begin{equation}
  a_{ti} =
  \begin{cases}
    W_i (e_t - e_i^{\mathrm{opt}})^2 + a_i^{\mathrm{max}},
      & \text{if } a_{ti} > 0\\
      0, & \text{if } a_{ti} \leq 0
  \end{cases}
\end{equation}

\noindent where $W_i$ controls the width of the curve for population $i$, $e_t$ represents the environmental value at generation $t$, $e_i^{\mathrm{opt}}$ represents the optimal environmental value for population $i$, and $a_i^{\mathrm{max}}$ represents the maximum possible $a$ value for population $i$. We set the $W_i$ parameters (Table \ref{t:pars}) and calculated the $a_i^{\mathrm{max}}$ parameters so that the area under each curve $A_i$ was equal. See Figure \ref{f:curves}a for example environmental-tolerance curves.

\subsubsection{Straying}

We implemented straying as in \citet{cooper1999}. We set up the metapopulation in a simple scenario: we arranged the populations in a line and those that were nearer to each other were more likely to stray between each other. Two parameters controlled the straying: the fraction of fish $f_{\mathrm{stray}}$ that stray from their natal stream in any generation and the rate $m$ at which this straying between streams decays with distance. We calculated the number of salmon straying from stream $j$ to stream $i$ as

\begin{equation}
  \mathrm{strays}_{tij} = f_{\mathrm{stray}} R_{tj}
    \frac{e^{-m \lvert i-j \rvert }}
      {\displaystyle\sum\limits_{\substack{k = 1 \\
    k \neq j}}^{n} e^{-m \lvert k-j \rvert }}
  \label{eq:stray}
\end{equation}

\noindent where $R_{tj}$ is the number of returning salmon at generation $t$ whose natal stream was stream $j$. The subscript $k$ represents a stream ID and $n$ the number of populations. The denominator is a normalizing constant to ensure the desired fraction of fish stray. See Figure \ref{f:stray} for an example straying matrix.

\subsection{Fishing}

Our simulation used a simple set of rules to establish escapement targets and harvest the fish. Every $f_\mathrm{assess}$ years (default of five years) our simulation fitted a spawner-return function and target harvest rate $H_{\mathrm{tar}}$ was set based on \citet{hilborn1992} as

\begin{equation}
  H_{\mathrm{tar}} = \frac{A}{b (0.5 - 0.07a)}
  \label{eq:esc}
\end{equation}

\noindent where $A$ represents the return abundance and $a$ and $b$ represent the Ricker model parameters. We included implementation uncertainty in the actual harvest rate $H_{\mathrm{act}}$ as

\begin{equation}
  H_{\mathrm{act}} = \mathrm{beta}(\alpha_h, \beta_h)
\end{equation}

\noindent where $\alpha_h$ and $\beta_h$ are the location and shape parameters in a beta distribution. They can be calculated from the desired mean $\mu_h$ and standard deviation $\sigma_h$ as \citep[p.~97]{morgan1990}

\begin{align}
  \alpha_h &= \mu_h^2
                \left(
                \frac{1 - \mu_h}{\sigma_h^2} - \frac{1}{\mu_h}
                \right)\\
   \beta_h &= \alpha \left({\frac{1}{\mu_h} - 1}\right).
\end{align}

\noindent Further, to establish a range of spawner-return values and to mimic the start of an open-access fishery, for the first 30 years we drew the fraction of fish harvested randomly from a uniform distribution between 0.1 and 0.9. We discarded these initial 30 years as a burn-in period throughout our analyses.

\subsection{Environmental dynamics}

We evaluated portfolio performance under short- and long-term environmental dynamics. We represented short-term dynamics as a stationary first-order autoregressive process, AR(1), with correlation $\rho_e$

\begin{equation}
  e_t = e_{t-1} \rho_e + d_t, d_t \sim \mathrm{N}(0, \sigma_d)
\end{equation}

\noindent where $e_t$ represents the environmental value in generation $t$ and $d$ represents normally distributed deviations of mean 0 and standard deviation $\sigma_d$. We represented long-term environmental dynamics as a linear shift in the environmental value through time

\begin{equation}
  e_t = \beta_e t - \overline{\beta_e t}
\end{equation}

\noindent where $\beta_e$ represents the slope. To maintain a balanced response, we centered the trend by subtracting the mean $\overline{\beta_e t}$ so that midway through the simulation (after any burn-in period) the environmental value was at the mean environmental tolerance.

\subsection{Conservation rules of thumb}

We evaluated two sets of conservation rules of thumb: (1) spatial response diversity conservation strategies in an idealistic scenario where you can detect response diversity, and (2) a more realistic scenario where we know little about response diversity and we're left with a choice of how many populations to conserve.

We evaluated four spatial conservation rules of thumb (Figure \ref{f:curves}b--e). In all spatial scenarios, we conserved four populations and set the unfished equilibrium biomass of the remaining populations to near elimination (five salmon). These reduced populations could still receive straying salmon but were unlikely to rebuild on their own to a substantial abundance. The four scenarios were:

\begin{enumerate}
\def\labelenumi{\arabic{enumi}.}
\itemsep1pt\parskip0pt\parsep0pt
\item
  Conserve an even sampling of response diversity.
\item
  Conserve the most stable populations only.
\item
  Conserve one half of the metapopulation.
\item
  Conserve the other half of the metapopulation.
\end{enumerate}

In reality we rarely know precise levels of response diversity. We therefore additionally considered a case where the conservation was randomly assigned with respect to response diversity but where different numbers of streams could be conserved. We considered conserving from two to 16 streams. Similarly to the spatial strategies, we reduced the capacity of the remaining streams to the nominal level of five salmon.

\section{Results}

\subsection{Which populations to conserve?}

\subsubsection{Short-term environment}

Given strong short-term environmental fluctuations, conserving response diversity buffers the risk properties of an ecological portfolio (Fig.~\ref{f:sp-mv}a). In our simulation, the median variance of generation-to-generation rate of change in abundance was X times lower given balanced response diversity (full range of responses or most stable only vs.~conserving one half or the other). In fact, even though by conserving the full range of responses, the portfolio was comprised of warm and cool-thriving populations that were more variable on their own, each was balanced by an opposing population. The portfolio risk was therefore comparable between the full range of responses and most stable only portfolios.

We can see the mechanism behind these portfolio properties by inspecting example population time series (Fig.~\ref{f:sp-mv}c, d). If only the upper or lower half of response diversity is conserved, the portfolio tends to do well or poorly depending on the environmental conditions (Fig.~\ref{f:sp-mv}d). This risk is buffered with balanced response diversity (Fig.~\ref{f:sp-mv}c).

\subsubsection{Long-term environment}

Given long-term environmental change, the choice of which populations to conserve affects the return properties of an ecological portfolio (Fig.~\ref{f:sp-mv}b). By conserving balance response diversity, an ecological manager is hedging his or her bets on what will happen with the environment and how the populations will respond. The typical return for a balanced response diversity strategy was zero --- the metapopulation neither increased or decreased in abundance in the long run. By conserving only the upper or lower half of response diversity, a conservation manager is putting all his or her eggs in one basket --- the metapopulation might do really well through time or it might do really poorly. The example metapopulation abundance time series (Fig.~\ref{f:sp-mv}d, f) illustrate this effect. By conserving response diversity, when one population is doing poorly, another is doing well and the metapopulation abundance remains stationary through time.

Notably, in theses simulations, if a managers invested in the populations that were doing well at the beginning they would have had the lowest rate-of-return portfolio in the end (purple portfolios in Fig.~\ref{f:sp-mv}b).

Spatial conservation strategies in the face of longterm environmental change hinge on whether you ``get it right'' --- whether you choose just the right populations to conserve.

\subsection{How many populations to conserve?}

Given a scenario where we don't know the distribution of population-level response diversity, portfolio optimization informs us about the risk buffering from maintaining multiple populations (Fig.~\ref{f:n-mv}). In short, investing in more populations buffers portfolio risk.

\subsubsection{Short-term environment}

Given short-term environmental noise, conserving more populations buffers portfolio risk while the random conservation of response diversity creates a spread of metapopulation risk for the same number of populations conserved (Fig.~\ref{f:n-mv}a). For example, a metapopulation with eight conserved populations is X times less risky than a metapopulation with only four. We can see this risk-buffering effect through example metapopulations in Fig.~\ref{f:n-mv}c and \ref{f:n-mv}d. We note that the risk-return axes of portfolio optimization ignore the absolute-abundance dimension (Fig.~\ref{f:n-mv}d). As one would expect, conserving fewer populations also results in lower-abundance metapopulations.

\subsubsection{Long-term environment}

Given long-term environmental noise, conserving more populations also buffers portfolio risk. However, in contrast to the short-term environmental noise scenario, the unknown response diversity creates a spread of possible metapopulation return for the same number of conserved populations (Fig.~\ref{f:n-mv}b). Here, the number of populations conserved buffers non-systematic (i.e.~not-environmentally-driven) stochasticity.

\section{Discussion}

\section{Acknowledgements}

\bibliographystyle{apalike}

\bibliography{jshort,ms}

\clearpage

\section{Tables}

\begin{table}[h!]
\centering
\small
\caption{Components of salmon metapopulation portfolios KEEP THIS?}
\begin{tabular}{p{3.6cm}p{7.5cm}}
\toprule
Component          & Definition for the salmon portfolio\\
\midrule
Assets             & Stream-level salmon populations; possibly a Viable Salmonid Population\\
Portfolio          & The salmon metapopulation; possibly an Evolutionarily Significant Unit\\
Portfolio managers & Salmon managers\\
Investors          & Salmon managers, conservation agency, or salmon fishers\\
Asset weights      & Carrying capacity (specifically the $b$ parameters in a Ricker model)\\
Asset returns      & Rate of change of generation-to-generation salmon metapopulation abundance\\
Asset risk         & Variance of generation-to-generation salmon metapopulation abundance\\
\bottomrule
\end{tabular}
\label{t:port}
\end{table}

\clearpage

\begin{table}[h!]
\centering
\footnotesize
\caption{Input parameters to the salmon metapopulation simulation with default values.}
\begin{tabular}{p{7.7cm}p{1.4cm}p{3.6cm}}
\toprule
Description                                                          & Symbol                & Value \\
\midrule

\textit{Population dynamics parameters}                              &                       & \\
Stock-recruit residual standard deviation (on log scale)             & $\sigma_r$            & 0.30  \\
AR(1) serial correlation of stock-recruit residuals                  & $\rho_w$              & 0.40  \\
Fraction of fish that stray from natal streams                       & $f_{\mathrm{stray}}$  & 0.02  \\
Exponential rate of decay of straying with distance                  & $m$                   & 0.3  \\

\noalign{\vskip 3mm}
\textit{Environmental parameters}                                    &                       & \\
Width of environmental-tolerance curves for populations $i$ 1 to $n$ & $W_i$                 & seq(0.05, 0.02, 0.02 0.05, length = n pop)\\
Optimum environmental value for populations $i$ 1 to $n$             & $e_i^{\mathrm{opt}}$  & seq(13, 19, length = n pop)\\
Area under each environmental-tolerance curve in environmental units & $A$                   & 30\\

Standard deviation of short-term environmental fluctuations          & $\sigma_d$            & 2 \\
AR(1) autocorrelation of short-term environmental fluctuations       & $\rho_e$              & 0.1 \\
Slope of long-term environmental signal                              & $\beta_e$             & 0.114 \\

\noalign{\vskip 3mm}
\textit{Fishery parameters}                                          &                       & \\
Standard deviation of beta distribution for implementation error     & $\sigma_{h}$          & 0.05  \\
Frequency of assessment (years)                                      & $f_{\mathrm{assess}}$ & 5  \\
\bottomrule
\end{tabular}
\label{t:pars}
\end{table}

\clearpage

\section{Figures}

\clearpage

\begin{figure}[htbp]
\centering
\includegraphics[height=5.5in]{../examples/simulation-diagram2.pdf}
\caption{Flow chart of the salmon-metapopulation simulation. There are $n$ salmon populations and $t$ generations. Blue text indicates values that are generated before the simulation progresses through time. Red text indicates steps in which calculations are performed through time. Black text indicates values that are calculated. Grey text indicates parameters that can be set. Green text indicates the looping structure of the simulation.}
\label{f:sim-flow}
\end{figure}

\clearpage

\begin{figure}[htbp]
\centering
\includegraphics[width=3.0in]{../examples/thermal-curve-scenarios.pdf}
\caption{Different ways of prioritizing response-diversity conservation. Panel a shows the thermal tolerance cures for ten possible populations and panels b--e show different ways of prioritizing four of those populations. The curves describe how productivity varies with the environment for a given population. Some populations thrive at low environmental values (cool colours) and some at high (warm colours) values. Some are tolerant to a wider range of environmental conditions (yellow-to-green colours) but with a lower maximum productivity. The total possible productivity (the area under the curves) is the same for each population.}
\label{f:curves}
\end{figure}

\clearpage

\begin{figure}[htbp]
\centering
\includegraphics[width=4.0in]{../examples/spatial-arma-sim.pdf}
\caption{The components of an example metapopulation simulation.  We show, from top to bottom, the environmental signal, the resulting productivity parameter (Ricker $a$), the salmon returns, fisheries catch, salmon escapement, salmon straying from their natal streams, salmon joining from other streams, stock-recruit residuals on a log scale, and the estimated $a$ and $b$ parameters in the fitted Ricker curve. The colored lines indicate populations that thrive at low (cool colours) to high (warm colours) environmental values.}
\label{f:sp-eg}
\end{figure}

\clearpage

\begin{figure}[htbp]
\centering
\includegraphics[width=5.0in]{../examples/spatial-mv.pdf}
\caption{The importance of preserving environmental response diversity through spatial conservation strategies. The conservation strategies correspond to Fig.~\ref{f:curves} and represent conserving a range of responses (green), the most stable populations only (orange), or one type of environmental response (purple and pink).  In risk-return space we show environmental scenarios that are comprised primarily of (a) short-term and (b) long-term environmental fluctuations (see Fig.~X). The dots show simulated metapopulations and the contours show 25\% and 75\% quantiles across 500 simulations per strategy. We also show example metapopulation abundance time series for the (c, e) short-term and (d, f) long-term  environmental-fluctuation scenarios.}
\label{f:sp-mv}
\end{figure}

\clearpage

\begin{figure}[htbp]
\centering
\includegraphics[width=5.0in]{../examples/cons-plans-n.pdf}
\caption{The importance of preserving as many subpopulations as possible when we don't know how response diversity is distributed. In risk-return space we show environmental scenarios that are comprised primarily of (a) short-term and (b) long-term environmental fluctuations (see Fig.~X). We show metapopulations in which 2 (red), 4 (orange), 8 (yellow), or 16 (green) populations of random response diversity are conserved. The dots show simulated metapopulations and the contours show 25\% and 75\% quantiles across 500 simulations per strategy. We also show example metapopulation (c) rate-of-change and (d) abundance time series for the short-term environmental-fluctuation scenario.}
\label{f:n-mv}
\end{figure}

\clearpage


\bibliographystyle{prsb}
%\bibliography{jshort,ms}
\documentclass[12pt]{article}
\usepackage{geometry}
\geometry{verbose,letterpaper,tmargin=2.54cm,bmargin=2.54cm,lmargin=2.54cm,rmargin=2.54cm} 
\geometry{letterpaper}
\usepackage{amssymb}
\usepackage{amsmath}
\usepackage{amsfonts}
\usepackage{graphicx}
\usepackage{setspace}
\usepackage[round]{natbib}
\bibpunct{(}{)}{;}{a}{}{;}
\textheight 22.0cm
\usepackage{lineno}
\usepackage{xcolor}
\renewcommand\linenumberfont{\normalfont\tiny\sffamily\color{gray}}
\usepackage{booktabs}
\usepackage{cite}

\newlabel{f:flowchart}{{1}{999}}
\newlabel{f:curves}{{2}{999}}
\newlabel{f:ts}{{3}{999}}
\newlabel{f:sp}{{4}{999}}
\newlabel{f:n}{{5}{999}}

\newcommand{\somR}{Appendix A}
\newcommand{\somparam}{Appendix B}
\newcommand{\somstray}{Appendix C}
\newcommand{\somsens}{Appendix D}
\newcommand{\somcor}{Appendix E}
\newcommand{\somts}{Appendix F}

% SOM:
\newlabel{t:pars}{{B1}{2}}
\newlabel{f:stray}{{B1}{3}}
\newlabel{f:eg-sens}{{D1}{4}}
\newlabel{f:ret-corr}{{E1}{5}}
\newlabel{f:eg-sp-arma-full}{{F1}{6}}
\newlabel{f:eg-sp-arma-half}{{F2}{7}}
\newlabel{f:eg-sp-linear-full}{{F3}{8}}
\newlabel{f:eg-sp-linear-half}{{F4}{9}}
\newlabel{f:eg-n-arma-two}{{F5}{10}}
\newlabel{f:eg-n-arma-sixteen}{{F6}{11}}
\newlabel{f:eg-n-linear-two}{{F7}{12}}
\newlabel{f:eg-n-linear-sixteen}{{F8}{13}}


\DeclareGraphicsExtensions{.pdf}

\widowpenalty=10000
\clubpenalty=10000

%\usepackage{titlesec}
%\titlespacing\section{0pt}{6pt plus 4pt minus 2pt}{-8pt plus 2pt minus 2pt}
%\titlespacing\subsection{0pt}{6pt plus 4pt minus 2pt}{-8pt plus 2pt minus 2pt}

\title{Portfolio conservation of metapopulations under\\climate change}

\author{
Sean C. Anderson$^{1\ast}$ \and Jonathan W. Moore$^{1,2}$ \and Michelle M. McClure$^3$ \and
Nicholas K. Dulvy$^1$ \and Andrew B. Cooper$^2$
}
\date{}

% remove numbers in front of sections:
\makeatletter
\renewcommand\@seccntformat[1]{}
\makeatother

\hyphenation{meta-pop-ulation meta-pop-ulations sub-pop-ulations sub-pop-ulation e-con-o-mist en-vi-ron-men-tal pri-or-i-ti-za-tion}

\begin{document}
\raggedright


\linenumbers

%\modulolinenumbers[2]
\begin{spacing}{1.9}
\setlength{\parindent}{0.9cm}

\maketitle

\emph{Other title ideas:}

\begin{itemize}
\item
  Prioritizing metapopulation conservation through the lens of portfolio theory
\item
  Portfolio optimization reveals\ldots{}
\item
  Portfolio theory reveals conservation strategies that\ldots{}
\item
  Intelligent tinkering with ecological portfolios: patterns of environmental response diversity drive salmon conservation priorities
\item
  Keeping every cog and wheel\ldots{}
\end{itemize}

\textbf{Ecology Letters}

\begin{itemize}
\itemsep1pt\parskip0pt\parsep0pt
\item
  letters: 5000 words, 6 figures or tables
\item
  ideas and perspectives: 7500 words, 10 figures (needs 300 word proposal)
\end{itemize}

\begin{quote}
Ecology Letters is a forum for the very rapid publication of the most novel research in ecology, research that is not yet in the public domain. Manuscripts relating to the ecology of all taxa, in any biome and geographic area will be considered, and priority will be given to those papers exploring or testing clearly stated hypotheses. The journal publishes concise papers that merit urgent publication by virtue of their originality, general interest and their contribution to new developments in ecology. We discourage purely descriptive papers and those merely confirming or extending results of previous work.
\end{quote}

\textbf{Ecological Applications}

\begin{itemize}
\itemsep1pt\parskip0pt\parsep0pt
\item
  60 pages including everything
\end{itemize}

\begin{quote}
Ecological Applications is concerned broadly with the applications of ecological science to environmental problems. It publishes papers that develop scientific principles to support environmental decision-making, as well as papers that discuss the application of ecological concepts to environmental issues, policy, and management. Papers may report on experimental tests, actual applications, scientific decision support techniques, economic analyses, social implications of environmental issues, or other relevant topics. Statistical or experimental methods papers that support research and applications are welcome. Papers submitted to Ecological Applications should be accessible to both scholars and practitioners.
\end{quote}

\section{Long and rambling abstract that's a bit old now}

\begin{enumerate}
\def\labelenumi{\arabic{enumi}.}
\item
  Managing risk is fundamental to the conservation of an endangered species. When an endangered species exists as a metapopulation, we can manage risk at two levels: at the population level or at the metapopulation level. Whereas risk is typically managed at the population level, a portfolio approach to managing risk might consider how conservation affects the ``weight'' of each population in a metapopulation ``portfolio''.
\item
  Here, we ask how a portfolio approach to managing risk can inform the spatial conservation of metapopulations in a changing world. To answer this, we develop a salmon metapopulation simulation in which population-specific productivity is driven by spatially-distributed environmental tolerance and patterns of short- and long-term environmental change. We then implement different spatial conservation ``rules of thumb'' that control the population-specific carrying capacities and evaluate the salmon portfolios along risk and return axes, similarly to how financial portfolios are assessed.
\item
  Our results show, first, that maintaining populations with a variety of environmental tolerances gives the best chance at an efficient ecological portfolio --- minimizing metapopulation variance while maximizing metapopulation growth rate. This finding emphasizes the risk of allowing large spatial blocks of habitat destruction, say through the development of dams. Second, we show that focusing on well-performing stocks now at the detriment of others is at best equivalent to a risky but efficient portfolio and is more likely a risky and inefficient portfolio --- it neither minimizes metapopulation variance nor maximizes growth rate compared to other strategies. Third, we show that maintaining more populations reduces metapopulation risk for the same spatial conservation strategy. Given a lack of knowledge of how populations respond to the environment, the most risk-averse approach is to conserve as many populations as possible.
\item
  Our findings highlight three key points: (1) the conservation priority of maintaining biocomplexity and therefore environmental response diversity, (2) the research priority of identifying differences in environmental tolerance given predicted environmental changes, and (3) the utility of considering risk for groups of fish stocks --- especially given environmental, biological, and implementation uncertainty --- through the lens of portfolio theory.
\end{enumerate}

\section{Introduction}

\begin{itemize}
\itemsep1pt\parskip0pt\parsep0pt
\item
  Managing risk is fundamental to the conservation of an endangered species.

  \begin{itemize}
  \itemsep1pt\parskip0pt\parsep0pt
  \item
    risk is usually assessed on a population-by-population basis
  \item
    for metapopulations, we can consider risk for components or risk for the whole
  \end{itemize}
\item
  The management of financial portfolios provides another way of considering risk for metapopulations.

  \begin{itemize}
  \itemsep1pt\parskip0pt\parsep0pt
  \item
    performance of weighted assets
  \item
    efficient frontier concept
  \item
    the parallels with ecological metapopulations
  \end{itemize}
\item
  Pacific salmon are an ideal model to consider under the framework of portfolio prioritization because of their metapopulation structure, their societal importance and threatened status, and the need to prioritize conservation efforts.

  \begin{itemize}
  \itemsep1pt\parskip0pt\parsep0pt
  \item
    portfolio-like structure

    \begin{itemize}
    \itemsep1pt\parskip0pt\parsep0pt
    \item
      documented metapopulation structure
    \item
      predators and fisheries often integrate across stocks
    \end{itemize}
  \item
    Highly threatened; highly valued; face many threats

    \begin{itemize}
    \itemsep1pt\parskip0pt\parsep0pt
    \item
      damns
    \item
      roads
    \item
      other habitat degradation
    \item
      logging
    \item
      fishing
    \item
      changing climate
    \end{itemize}
  \item
    There's a need for prioritization

    \begin{itemize}
    \itemsep1pt\parskip0pt\parsep0pt
    \item
      because of the highly threatened nature
    \item
      conflicting goals for resource use
    \item
      limited resources
    \item
      Existing prioritization schemes
    \end{itemize}
  \end{itemize}
\item
  Portfolio stabilization through systematic asynchrony in salmon population dynamics can be driven by stocks responding uniquely to the same environment (`response diversity') or by physical features of the environment causing stocks to experience different environmental forces (`environmental filter').

  \begin{itemize}
  \itemsep1pt\parskip0pt\parsep0pt
  \item
    different responses to the environment: response diversity / biocomplexity

    \begin{itemize}
    \itemsep1pt\parskip0pt\parsep0pt
    \item
      genetic variation by stock
    \item
      can be driven through historical conditions
    \item
      we focus on this here
    \end{itemize}
  \item
    experiencing different environment: environmental filter concept (Moran like)

    \begin{itemize}
    \itemsep1pt\parskip0pt\parsep0pt
    \item
      spatially spread out, geographical and hydrological features filter the environment to create a variety of forces
    \item
      Typically a combination of both
    \end{itemize}
  \item
    Conversely, forces acting against asynchrony

    \begin{itemize}
    \itemsep1pt\parskip0pt\parsep0pt
    \item
      inverse: loss of population diversity
    \item
      inverse: moran
    \item
      but also: dispersal
    \item
      and: trophic interactions with synchronized species
    \end{itemize}
  \end{itemize}
\end{itemize}

Here, we ask how a portfolio approach to management can inform the conservation of metapopulations in a changing world. We ask two primary questions: (1) What does portfolio optimization tell us about different spatial approaches to prioritizing metapopulation conservation assuming that response diversity is spatially distributed? (2) If we don't know how response diversity is distribute, what does portfolio optimization tell us about how many populations we should conserve? To answer these questions, we develop a salmon metapopulation simulation in which population-specific productivity is driven by spatially-distributed environmental tolerance and patterns of short- and long-term climatic change. We then implement different conservation ``rules of thumb'' that control the population-level carrying capacities and evaluate the salmon portfolios along risk and return axes, as a portfolio manager would. Our findings illustrate that: (1) conserving response diversity buffers metapopulation risk given short-term climate forcing and rate of return given long-term climate forcing, (2) conserving more subpopulations buffers risk regardless of the climate trend, and (3) considering metapopulations through the lens of portfolio theory provides a useful additional dimension through which we can evaluate conservation strategies.

\section{Methods}

We developed a salmon metapopulation simulation model that includes a stock-recruit relationship, demographic stochasticity, straying between populations, varying responses to the environment, escapement target setting, and implementation uncertainty. Under two kinds of environmental regimes we tested different conservation rules of thumb and evaluated these plans in risk-return space similar to how financial managers evaluate financial portfolios. See Figure \ref{f:sim-flow} for an illustration of the overall simulation structure. See Table \ref{t:pars} for a list of the input parameters to our simulation and their default values. We provide a package \texttt{metafolio} for the statistical software \texttt{R} \citep{r2013} as an appendix, to carry out the simulations and analyses we describe in this paper.

\subsection{Defining the ecological portfolio}

In our ecological portfolios, we defined assets as stream-level populations and the portfolios as salmon metapopulations (Table \ref{t:port}). We use the terms \emph{stream} and \emph{populations} interchangeably to represent the portfolio assets. We defined the portfolio investors as the stakeholders in the fishery and metapopulation performance. For example, we could consider fisheries managers, conservation agencies, or First Nations groups as investors. We defined asset value as the abundance of returning salmon in each stream and value of the portfolio as the overall metapopulation abundance. In this scenario, the equivalent to financial rate of return is the generation-to-generation rate of change of metapopulation abundance. We defined the financial asset investment weights as the capacity of the stream populations --- specifically the unfinished equilibrium stock size --- since maintaining or restoring habitat requires money, time, and resources. Investment in a population therefore represents investing in salmon habitat conservation or reconstruction.

\subsection{Salmon metapopulation dynamics}

The salmon metapopulation dynamics in our simulation were governed by a spawner-return relationship with demographic stochasticity and by straying between populations.

\subsubsection{Spawner-return relationship}

We defined the spawner-return relationship with a Ricker model \citep{ricker1954},

\begin{equation}
R_{ti} = S_{ti}e^{a_{ti}(1-S_{ti}/b_i) + w_{ti}}
\end{equation}

\noindent where $t$ represents a generation time, $i$ represents a population, $R$ is the number of returns, $S$ is the number of spawners, $a$ is the productivity parameter (which can vary with the environment), and $b$ is the density-dependent term (which is used as the asset weights in the portfolios). The term $w_{ti}$ represents first-order autocorrelated error. Formally, $w_{ti} = w_{ti-1} \rho_w + r_{ti}$, where $r_{ti}$ represents independent and normally distributed error with mean 0 and standard deviation of $\sigma_r$. The parameter $\rho_w$ represents the correlation between residuals from subsequent generations.

We manipulated the capacity and productivity parameters $b$ and $a$ as part of the portfolio simulation. The capacity parameters $b_i$ were controlled by the investment weights in the populations. For example, a large investment in a stream was represented by a larger unfished equilibrium stock size $b$ for stream $i$. The productivity parameters $a_{ti}$ were controlled by the interaction between an environmental signal and the stream-level population environmental-tolerance curves.

We generated the environmental-tolerance parabolas according to

\begin{equation}
  a_{ti} =
  \begin{cases}
    W_i (e_t - e_i^{\mathrm{opt}})^2 + a_i^{\mathrm{max}},
      & \text{if } a_{ti} > 0\\
      0, & \text{if } a_{ti} \leq 0
  \end{cases}
\end{equation}

\noindent where $W_i$ controls the width of the curve for population $i$, $e_t$ represents the environmental value at generation $t$, $e_i^{\mathrm{opt}}$ represents the optimal environmental value for population $i$, and $a_i^{\mathrm{max}}$ represents the maximum possible $a$ value for population $i$. We set the $W_i$ parameters (Table \ref{t:pars}) and calculated the $a_i^{\mathrm{max}}$ parameters so that the area under each curve $A_i$ was equal. See Figure \ref{f:curves}a for example environmental-tolerance curves.

\subsubsection{Straying}

We implemented straying as in \citet{cooper1999}. We set up the metapopulation in a simple scenario: we arranged the populations in a line and those that were nearer to each other were more likely to stray between each other. Two parameters controlled the straying: the fraction of fish $f_{\mathrm{stray}}$ that stray from their natal stream in any generation and the rate $m$ at which this straying between streams decays with distance. We calculated the number of salmon straying from stream $j$ to stream $i$ as

\begin{equation}
  \mathrm{strays}_{tij} = f_{\mathrm{stray}} R_{tj}
    \frac{e^{-m \lvert i-j \rvert }}
      {\displaystyle\sum\limits_{\substack{k = 1 \\
    k \neq j}}^{n} e^{-m \lvert k-j \rvert }}
  \label{eq:stray}
\end{equation}

\noindent where $R_{tj}$ is the number of returning salmon at generation $t$ whose natal stream was stream $j$. The subscript $k$ represents a stream ID and $n$ the number of populations. The denominator is a normalizing constant to ensure the desired fraction of fish stray. See Figure \ref{f:stray} for an example straying matrix.

\subsection{Fishing}

Our simulation used a simple set of rules to establish escapement targets and harvest the fish. Every $f_\mathrm{assess}$ years (default of five years) our simulation fitted a spawner-return function and target harvest rate $H_{\mathrm{tar}}$ was set based on \citet{hilborn1992} as

\begin{equation}
  H_{\mathrm{tar}} = \frac{A}{b (0.5 - 0.07a)}
  \label{eq:esc}
\end{equation}

\noindent where $A$ represents the return abundance and $a$ and $b$ represent the Ricker model parameters. We included implementation uncertainty in the actual harvest rate $H_{\mathrm{act}}$ as

\begin{equation}
  H_{\mathrm{act}} = \mathrm{beta}(\alpha_h, \beta_h)
\end{equation}

\noindent where $\alpha_h$ and $\beta_h$ are the location and shape parameters in a beta distribution. They can be calculated from the desired mean $\mu_h$ and standard deviation $\sigma_h$ as \citep[p.~97]{morgan1990}

\begin{align}
  \alpha_h &= \mu_h^2
                \left(
                \frac{1 - \mu_h}{\sigma_h^2} - \frac{1}{\mu_h}
                \right)\\
   \beta_h &= \alpha \left({\frac{1}{\mu_h} - 1}\right).
\end{align}

\noindent Further, to establish a range of spawner-return values and to mimic the start of an open-access fishery, for the first 30 years we drew the fraction of fish harvested randomly from a uniform distribution between 0.1 and 0.9. We discarded these initial 30 years as a burn-in period throughout our analyses.

\subsection{Environmental dynamics}

We evaluated portfolio performance under short- and long-term environmental dynamics. We represented short-term dynamics as a stationary first-order autoregressive process, AR(1), with correlation $\rho_e$

\begin{equation}
  e_t = e_{t-1} \rho_e + d_t, d_t \sim \mathrm{N}(0, \sigma_d)
\end{equation}

\noindent where $e_t$ represents the environmental value in generation $t$ and $d$ represents normally distributed deviations of mean 0 and standard deviation $\sigma_d$. We represented long-term environmental dynamics as a linear shift in the environmental value through time

\begin{equation}
  e_t = \beta_e t - \overline{\beta_e t}
\end{equation}

\noindent where $\beta_e$ represents the slope. To maintain a balanced response, we centered the trend by subtracting the mean $\overline{\beta_e t}$ so that midway through the simulation (after any burn-in period) the environmental value was at the mean environmental tolerance.

\subsection{Conservation rules of thumb}

We evaluated two sets of conservation rules of thumb: (1) spatial response diversity conservation strategies in an idealistic scenario where you can detect response diversity, and (2) a more realistic scenario where we know little about response diversity and we're left with a choice of how many populations to conserve.

We evaluated four spatial conservation rules of thumb (Figure \ref{f:curves}b--e). In all spatial scenarios, we conserved four populations and set the unfished equilibrium biomass of the remaining populations to near elimination (five salmon). These reduced populations could still receive straying salmon but were unlikely to rebuild on their own to a substantial abundance. The four scenarios were:

\begin{enumerate}
\def\labelenumi{\arabic{enumi}.}
\itemsep1pt\parskip0pt\parsep0pt
\item
  Conserve an even sampling of response diversity.
\item
  Conserve the most stable populations only.
\item
  Conserve one half of the metapopulation.
\item
  Conserve the other half of the metapopulation.
\end{enumerate}

In reality we rarely know precise levels of response diversity. We therefore additionally considered a case where the conservation was randomly assigned with respect to response diversity but where different numbers of streams could be conserved. We considered conserving from two to 16 streams. Similarly to the spatial strategies, we reduced the capacity of the remaining streams to the nominal level of five salmon.

\section{Results}

\subsection{Which populations to conserve?}

\subsubsection{Short-term environment}

Given strong short-term environmental fluctuations, conserving response diversity buffers the risk properties of an ecological portfolio (Fig.~\ref{f:sp-mv}a). In our simulation, the median variance of generation-to-generation rate of change in abundance was X times lower given balanced response diversity (full range of responses or most stable only vs.~conserving one half or the other). In fact, even though by conserving the full range of responses, the portfolio was comprised of warm and cool-thriving populations that were more variable on their own, each was balanced by an opposing population. The portfolio risk was therefore comparable between the full range of responses and most stable only portfolios.

We can see the mechanism behind these portfolio properties by inspecting example population time series (Fig.~\ref{f:sp-mv}c, d). If only the upper or lower half of response diversity is conserved, the portfolio tends to do well or poorly depending on the environmental conditions (Fig.~\ref{f:sp-mv}d). This risk is buffered with balanced response diversity (Fig.~\ref{f:sp-mv}c).

\subsubsection{Long-term environment}

Given long-term environmental change, the choice of which populations to conserve affects the return properties of an ecological portfolio (Fig.~\ref{f:sp-mv}b). By conserving balance response diversity, an ecological manager is hedging his or her bets on what will happen with the environment and how the populations will respond. The typical return for a balanced response diversity strategy was zero --- the metapopulation neither increased or decreased in abundance in the long run. By conserving only the upper or lower half of response diversity, a conservation manager is putting all his or her eggs in one basket --- the metapopulation might do really well through time or it might do really poorly. The example metapopulation abundance time series (Fig.~\ref{f:sp-mv}d, f) illustrate this effect. By conserving response diversity, when one population is doing poorly, another is doing well and the metapopulation abundance remains stationary through time.

Notably, in theses simulations, if a managers invested in the populations that were doing well at the beginning they would have had the lowest rate-of-return portfolio in the end (purple portfolios in Fig.~\ref{f:sp-mv}b).

Spatial conservation strategies in the face of longterm environmental change hinge on whether you ``get it right'' --- whether you choose just the right populations to conserve.

\subsection{How many populations to conserve?}

Given a scenario where we don't know the distribution of population-level response diversity, portfolio optimization informs us about the risk buffering from maintaining multiple populations (Fig.~\ref{f:n-mv}). In short, investing in more populations buffers portfolio risk.

\subsubsection{Short-term environment}

Given short-term environmental noise, conserving more populations buffers portfolio risk while the random conservation of response diversity creates a spread of metapopulation risk for the same number of populations conserved (Fig.~\ref{f:n-mv}a). For example, a metapopulation with eight conserved populations is X times less risky than a metapopulation with only four. We can see this risk-buffering effect through example metapopulations in Fig.~\ref{f:n-mv}c and \ref{f:n-mv}d. We note that the risk-return axes of portfolio optimization ignore the absolute-abundance dimension (Fig.~\ref{f:n-mv}d). As one would expect, conserving fewer populations also results in lower-abundance metapopulations.

\subsubsection{Long-term environment}

Given long-term environmental noise, conserving more populations also buffers portfolio risk. However, in contrast to the short-term environmental noise scenario, the unknown response diversity creates a spread of possible metapopulation return for the same number of conserved populations (Fig.~\ref{f:n-mv}b). Here, the number of populations conserved buffers non-systematic (i.e.~not-environmentally-driven) stochasticity.

\section{Discussion}

\section{Acknowledgements}

\bibliographystyle{apalike}

\bibliography{jshort,ms}

\clearpage

\section{Tables}

\begin{table}[h!]
\centering
\small
\caption{Components of salmon metapopulation portfolios KEEP THIS?}
\begin{tabular}{p{3.6cm}p{7.5cm}}
\toprule
Component          & Definition for the salmon portfolio\\
\midrule
Assets             & Stream-level salmon populations; possibly a Viable Salmonid Population\\
Portfolio          & The salmon metapopulation; possibly an Evolutionarily Significant Unit\\
Portfolio managers & Salmon managers\\
Investors          & Salmon managers, conservation agency, or salmon fishers\\
Asset weights      & Carrying capacity (specifically the $b$ parameters in a Ricker model)\\
Asset returns      & Rate of change of generation-to-generation salmon metapopulation abundance\\
Asset risk         & Variance of generation-to-generation salmon metapopulation abundance\\
\bottomrule
\end{tabular}
\label{t:port}
\end{table}

\clearpage

\begin{table}[h!]
\centering
\footnotesize
\caption{Input parameters to the salmon metapopulation simulation with default values.}
\begin{tabular}{p{7.7cm}p{1.4cm}p{3.6cm}}
\toprule
Description                                                          & Symbol                & Value \\
\midrule

\textit{Population dynamics parameters}                              &                       & \\
Stock-recruit residual standard deviation (on log scale)             & $\sigma_r$            & 0.30  \\
AR(1) serial correlation of stock-recruit residuals                  & $\rho_w$              & 0.40  \\
Fraction of fish that stray from natal streams                       & $f_{\mathrm{stray}}$  & 0.02  \\
Exponential rate of decay of straying with distance                  & $m$                   & 0.3  \\

\noalign{\vskip 3mm}
\textit{Environmental parameters}                                    &                       & \\
Width of environmental-tolerance curves for populations $i$ 1 to $n$ & $W_i$                 & seq(0.05, 0.02, 0.02 0.05, length = n pop)\\
Optimum environmental value for populations $i$ 1 to $n$             & $e_i^{\mathrm{opt}}$  & seq(13, 19, length = n pop)\\
Area under each environmental-tolerance curve in environmental units & $A$                   & 30\\

Standard deviation of short-term environmental fluctuations          & $\sigma_d$            & 2 \\
AR(1) autocorrelation of short-term environmental fluctuations       & $\rho_e$              & 0.1 \\
Slope of long-term environmental signal                              & $\beta_e$             & 0.114 \\

\noalign{\vskip 3mm}
\textit{Fishery parameters}                                          &                       & \\
Standard deviation of beta distribution for implementation error     & $\sigma_{h}$          & 0.05  \\
Frequency of assessment (years)                                      & $f_{\mathrm{assess}}$ & 5  \\
\bottomrule
\end{tabular}
\label{t:pars}
\end{table}

\clearpage

\section{Figures}

\clearpage

\begin{figure}[htbp]
\centering
\includegraphics[height=5.5in]{../examples/simulation-diagram2.pdf}
\caption{Flow chart of the salmon-metapopulation simulation. There are $n$ salmon populations and $t$ generations. Blue text indicates values that are generated before the simulation progresses through time. Red text indicates steps in which calculations are performed through time. Black text indicates values that are calculated. Grey text indicates parameters that can be set. Green text indicates the looping structure of the simulation.}
\label{f:sim-flow}
\end{figure}

\clearpage

\begin{figure}[htbp]
\centering
\includegraphics[width=3.0in]{../examples/thermal-curve-scenarios.pdf}
\caption{Different ways of prioritizing response-diversity conservation. Panel a shows the thermal tolerance cures for ten possible populations and panels b--e show different ways of prioritizing four of those populations. The curves describe how productivity varies with the environment for a given population. Some populations thrive at low environmental values (cool colours) and some at high (warm colours) values. Some are tolerant to a wider range of environmental conditions (yellow-to-green colours) but with a lower maximum productivity. The total possible productivity (the area under the curves) is the same for each population.}
\label{f:curves}
\end{figure}

\clearpage

\begin{figure}[htbp]
\centering
\includegraphics[width=4.0in]{../examples/spatial-arma-sim.pdf}
\caption{The components of an example metapopulation simulation.  We show, from top to bottom, the environmental signal, the resulting productivity parameter (Ricker $a$), the salmon returns, fisheries catch, salmon escapement, salmon straying from their natal streams, salmon joining from other streams, stock-recruit residuals on a log scale, and the estimated $a$ and $b$ parameters in the fitted Ricker curve. The colored lines indicate populations that thrive at low (cool colours) to high (warm colours) environmental values.}
\label{f:sp-eg}
\end{figure}

\clearpage

\begin{figure}[htbp]
\centering
\includegraphics[width=5.0in]{../examples/spatial-mv.pdf}
\caption{The importance of preserving environmental response diversity through spatial conservation strategies. The conservation strategies correspond to Fig.~\ref{f:curves} and represent conserving a range of responses (green), the most stable populations only (orange), or one type of environmental response (purple and pink).  In risk-return space we show environmental scenarios that are comprised primarily of (a) short-term and (b) long-term environmental fluctuations (see Fig.~X). The dots show simulated metapopulations and the contours show 25\% and 75\% quantiles across 500 simulations per strategy. We also show example metapopulation abundance time series for the (c, e) short-term and (d, f) long-term  environmental-fluctuation scenarios.}
\label{f:sp-mv}
\end{figure}

\clearpage

\begin{figure}[htbp]
\centering
\includegraphics[width=5.0in]{../examples/cons-plans-n.pdf}
\caption{The importance of preserving as many subpopulations as possible when we don't know how response diversity is distributed. In risk-return space we show environmental scenarios that are comprised primarily of (a) short-term and (b) long-term environmental fluctuations (see Fig.~X). We show metapopulations in which 2 (red), 4 (orange), 8 (yellow), or 16 (green) populations of random response diversity are conserved. The dots show simulated metapopulations and the contours show 25\% and 75\% quantiles across 500 simulations per strategy. We also show example metapopulation (c) rate-of-change and (d) abundance time series for the short-term environmental-fluctuation scenario.}
\label{f:n-mv}
\end{figure}

\clearpage


\bibliographystyle{ecology3}
\bibliography{jshort,ms}
%\documentclass[12pt]{article}
\usepackage{geometry}
\geometry{verbose,letterpaper,tmargin=2.54cm,bmargin=2.54cm,lmargin=2.54cm,rmargin=2.54cm} 
\geometry{letterpaper}
\usepackage{amssymb}
\usepackage{amsmath}
\usepackage{amsfonts}
\usepackage{graphicx}
\usepackage{setspace}
\usepackage[round]{natbib}
\bibpunct{(}{)}{;}{a}{}{;}
\textheight 22.0cm
\usepackage{lineno}
\usepackage{xcolor}
\renewcommand\linenumberfont{\normalfont\tiny\sffamily\color{gray}}
\usepackage{booktabs}
\usepackage{cite}

\newlabel{f:flowchart}{{1}{999}}
\newlabel{f:curves}{{2}{999}}
\newlabel{f:ts}{{3}{999}}
\newlabel{f:sp}{{4}{999}}
\newlabel{f:n}{{5}{999}}

\newcommand{\somR}{Appendix A}
\newcommand{\somparam}{Appendix B}
\newcommand{\somstray}{Appendix C}
\newcommand{\somsens}{Appendix D}
\newcommand{\somcor}{Appendix E}
\newcommand{\somts}{Appendix F}

% SOM:
\newlabel{t:pars}{{B1}{2}}
\newlabel{f:stray}{{B1}{3}}
\newlabel{f:eg-sens}{{D1}{4}}
\newlabel{f:ret-corr}{{E1}{5}}
\newlabel{f:eg-sp-arma-full}{{F1}{6}}
\newlabel{f:eg-sp-arma-half}{{F2}{7}}
\newlabel{f:eg-sp-linear-full}{{F3}{8}}
\newlabel{f:eg-sp-linear-half}{{F4}{9}}
\newlabel{f:eg-n-arma-two}{{F5}{10}}
\newlabel{f:eg-n-arma-sixteen}{{F6}{11}}
\newlabel{f:eg-n-linear-two}{{F7}{12}}
\newlabel{f:eg-n-linear-sixteen}{{F8}{13}}


\DeclareGraphicsExtensions{.pdf}

\widowpenalty=10000
\clubpenalty=10000

%\usepackage{titlesec}
%\titlespacing\section{0pt}{6pt plus 4pt minus 2pt}{-8pt plus 2pt minus 2pt}
%\titlespacing\subsection{0pt}{6pt plus 4pt minus 2pt}{-8pt plus 2pt minus 2pt}

\title{Portfolio conservation of metapopulations under\\climate change}

\author{
Sean C. Anderson$^{1\ast}$ \and Jonathan W. Moore$^{1,2}$ \and Michelle M. McClure$^3$ \and
Nicholas K. Dulvy$^1$ \and Andrew B. Cooper$^2$
}
\date{}

% remove numbers in front of sections:
\makeatletter
\renewcommand\@seccntformat[1]{}
\makeatother

\hyphenation{meta-pop-ulation meta-pop-ulations sub-pop-ulations sub-pop-ulation e-con-o-mist en-vi-ron-men-tal pri-or-i-ti-za-tion}

\begin{document}
\raggedright


\linenumbers

%\modulolinenumbers[2]
\begin{spacing}{1.9}
\setlength{\parindent}{0.9cm}

\maketitle

\emph{Other title ideas:}

\begin{itemize}
\item
  Prioritizing metapopulation conservation through the lens of portfolio theory
\item
  Portfolio optimization reveals\ldots{}
\item
  Portfolio theory reveals conservation strategies that\ldots{}
\item
  Intelligent tinkering with ecological portfolios: patterns of environmental response diversity drive salmon conservation priorities
\item
  Keeping every cog and wheel\ldots{}
\end{itemize}

\textbf{Ecology Letters}

\begin{itemize}
\itemsep1pt\parskip0pt\parsep0pt
\item
  letters: 5000 words, 6 figures or tables
\item
  ideas and perspectives: 7500 words, 10 figures (needs 300 word proposal)
\end{itemize}

\begin{quote}
Ecology Letters is a forum for the very rapid publication of the most novel research in ecology, research that is not yet in the public domain. Manuscripts relating to the ecology of all taxa, in any biome and geographic area will be considered, and priority will be given to those papers exploring or testing clearly stated hypotheses. The journal publishes concise papers that merit urgent publication by virtue of their originality, general interest and their contribution to new developments in ecology. We discourage purely descriptive papers and those merely confirming or extending results of previous work.
\end{quote}

\textbf{Ecological Applications}

\begin{itemize}
\itemsep1pt\parskip0pt\parsep0pt
\item
  60 pages including everything
\end{itemize}

\begin{quote}
Ecological Applications is concerned broadly with the applications of ecological science to environmental problems. It publishes papers that develop scientific principles to support environmental decision-making, as well as papers that discuss the application of ecological concepts to environmental issues, policy, and management. Papers may report on experimental tests, actual applications, scientific decision support techniques, economic analyses, social implications of environmental issues, or other relevant topics. Statistical or experimental methods papers that support research and applications are welcome. Papers submitted to Ecological Applications should be accessible to both scholars and practitioners.
\end{quote}

\section{Long and rambling abstract that's a bit old now}

\begin{enumerate}
\def\labelenumi{\arabic{enumi}.}
\item
  Managing risk is fundamental to the conservation of an endangered species. When an endangered species exists as a metapopulation, we can manage risk at two levels: at the population level or at the metapopulation level. Whereas risk is typically managed at the population level, a portfolio approach to managing risk might consider how conservation affects the ``weight'' of each population in a metapopulation ``portfolio''.
\item
  Here, we ask how a portfolio approach to managing risk can inform the spatial conservation of metapopulations in a changing world. To answer this, we develop a salmon metapopulation simulation in which population-specific productivity is driven by spatially-distributed environmental tolerance and patterns of short- and long-term environmental change. We then implement different spatial conservation ``rules of thumb'' that control the population-specific carrying capacities and evaluate the salmon portfolios along risk and return axes, similarly to how financial portfolios are assessed.
\item
  Our results show, first, that maintaining populations with a variety of environmental tolerances gives the best chance at an efficient ecological portfolio --- minimizing metapopulation variance while maximizing metapopulation growth rate. This finding emphasizes the risk of allowing large spatial blocks of habitat destruction, say through the development of dams. Second, we show that focusing on well-performing stocks now at the detriment of others is at best equivalent to a risky but efficient portfolio and is more likely a risky and inefficient portfolio --- it neither minimizes metapopulation variance nor maximizes growth rate compared to other strategies. Third, we show that maintaining more populations reduces metapopulation risk for the same spatial conservation strategy. Given a lack of knowledge of how populations respond to the environment, the most risk-averse approach is to conserve as many populations as possible.
\item
  Our findings highlight three key points: (1) the conservation priority of maintaining biocomplexity and therefore environmental response diversity, (2) the research priority of identifying differences in environmental tolerance given predicted environmental changes, and (3) the utility of considering risk for groups of fish stocks --- especially given environmental, biological, and implementation uncertainty --- through the lens of portfolio theory.
\end{enumerate}

\section{Introduction}

\begin{itemize}
\itemsep1pt\parskip0pt\parsep0pt
\item
  Managing risk is fundamental to the conservation of an endangered species.

  \begin{itemize}
  \itemsep1pt\parskip0pt\parsep0pt
  \item
    risk is usually assessed on a population-by-population basis
  \item
    for metapopulations, we can consider risk for components or risk for the whole
  \end{itemize}
\item
  The management of financial portfolios provides another way of considering risk for metapopulations.

  \begin{itemize}
  \itemsep1pt\parskip0pt\parsep0pt
  \item
    performance of weighted assets
  \item
    efficient frontier concept
  \item
    the parallels with ecological metapopulations
  \end{itemize}
\item
  Pacific salmon are an ideal model to consider under the framework of portfolio prioritization because of their metapopulation structure, their societal importance and threatened status, and the need to prioritize conservation efforts.

  \begin{itemize}
  \itemsep1pt\parskip0pt\parsep0pt
  \item
    portfolio-like structure

    \begin{itemize}
    \itemsep1pt\parskip0pt\parsep0pt
    \item
      documented metapopulation structure
    \item
      predators and fisheries often integrate across stocks
    \end{itemize}
  \item
    Highly threatened; highly valued; face many threats

    \begin{itemize}
    \itemsep1pt\parskip0pt\parsep0pt
    \item
      damns
    \item
      roads
    \item
      other habitat degradation
    \item
      logging
    \item
      fishing
    \item
      changing climate
    \end{itemize}
  \item
    There's a need for prioritization

    \begin{itemize}
    \itemsep1pt\parskip0pt\parsep0pt
    \item
      because of the highly threatened nature
    \item
      conflicting goals for resource use
    \item
      limited resources
    \item
      Existing prioritization schemes
    \end{itemize}
  \end{itemize}
\item
  Portfolio stabilization through systematic asynchrony in salmon population dynamics can be driven by stocks responding uniquely to the same environment (`response diversity') or by physical features of the environment causing stocks to experience different environmental forces (`environmental filter').

  \begin{itemize}
  \itemsep1pt\parskip0pt\parsep0pt
  \item
    different responses to the environment: response diversity / biocomplexity

    \begin{itemize}
    \itemsep1pt\parskip0pt\parsep0pt
    \item
      genetic variation by stock
    \item
      can be driven through historical conditions
    \item
      we focus on this here
    \end{itemize}
  \item
    experiencing different environment: environmental filter concept (Moran like)

    \begin{itemize}
    \itemsep1pt\parskip0pt\parsep0pt
    \item
      spatially spread out, geographical and hydrological features filter the environment to create a variety of forces
    \item
      Typically a combination of both
    \end{itemize}
  \item
    Conversely, forces acting against asynchrony

    \begin{itemize}
    \itemsep1pt\parskip0pt\parsep0pt
    \item
      inverse: loss of population diversity
    \item
      inverse: moran
    \item
      but also: dispersal
    \item
      and: trophic interactions with synchronized species
    \end{itemize}
  \end{itemize}
\end{itemize}

Here, we ask how a portfolio approach to management can inform the conservation of metapopulations in a changing world. We ask two primary questions: (1) What does portfolio optimization tell us about different spatial approaches to prioritizing metapopulation conservation assuming that response diversity is spatially distributed? (2) If we don't know how response diversity is distribute, what does portfolio optimization tell us about how many populations we should conserve? To answer these questions, we develop a salmon metapopulation simulation in which population-specific productivity is driven by spatially-distributed environmental tolerance and patterns of short- and long-term climatic change. We then implement different conservation ``rules of thumb'' that control the population-level carrying capacities and evaluate the salmon portfolios along risk and return axes, as a portfolio manager would. Our findings illustrate that: (1) conserving response diversity buffers metapopulation risk given short-term climate forcing and rate of return given long-term climate forcing, (2) conserving more subpopulations buffers risk regardless of the climate trend, and (3) considering metapopulations through the lens of portfolio theory provides a useful additional dimension through which we can evaluate conservation strategies.

\section{Methods}

We developed a salmon metapopulation simulation model that includes a stock-recruit relationship, demographic stochasticity, straying between populations, varying responses to the environment, escapement target setting, and implementation uncertainty. Under two kinds of environmental regimes we tested different conservation rules of thumb and evaluated these plans in risk-return space similar to how financial managers evaluate financial portfolios. See Figure \ref{f:sim-flow} for an illustration of the overall simulation structure. See Table \ref{t:pars} for a list of the input parameters to our simulation and their default values. We provide a package \texttt{metafolio} for the statistical software \texttt{R} \citep{r2013} as an appendix, to carry out the simulations and analyses we describe in this paper.

\subsection{Defining the ecological portfolio}

In our ecological portfolios, we defined assets as stream-level populations and the portfolios as salmon metapopulations (Table \ref{t:port}). We use the terms \emph{stream} and \emph{populations} interchangeably to represent the portfolio assets. We defined the portfolio investors as the stakeholders in the fishery and metapopulation performance. For example, we could consider fisheries managers, conservation agencies, or First Nations groups as investors. We defined asset value as the abundance of returning salmon in each stream and value of the portfolio as the overall metapopulation abundance. In this scenario, the equivalent to financial rate of return is the generation-to-generation rate of change of metapopulation abundance. We defined the financial asset investment weights as the capacity of the stream populations --- specifically the unfinished equilibrium stock size --- since maintaining or restoring habitat requires money, time, and resources. Investment in a population therefore represents investing in salmon habitat conservation or reconstruction.

\subsection{Salmon metapopulation dynamics}

The salmon metapopulation dynamics in our simulation were governed by a spawner-return relationship with demographic stochasticity and by straying between populations.

\subsubsection{Spawner-return relationship}

We defined the spawner-return relationship with a Ricker model \citep{ricker1954},

\begin{equation}
R_{ti} = S_{ti}e^{a_{ti}(1-S_{ti}/b_i) + w_{ti}}
\end{equation}

\noindent where $t$ represents a generation time, $i$ represents a population, $R$ is the number of returns, $S$ is the number of spawners, $a$ is the productivity parameter (which can vary with the environment), and $b$ is the density-dependent term (which is used as the asset weights in the portfolios). The term $w_{ti}$ represents first-order autocorrelated error. Formally, $w_{ti} = w_{ti-1} \rho_w + r_{ti}$, where $r_{ti}$ represents independent and normally distributed error with mean 0 and standard deviation of $\sigma_r$. The parameter $\rho_w$ represents the correlation between residuals from subsequent generations.

We manipulated the capacity and productivity parameters $b$ and $a$ as part of the portfolio simulation. The capacity parameters $b_i$ were controlled by the investment weights in the populations. For example, a large investment in a stream was represented by a larger unfished equilibrium stock size $b$ for stream $i$. The productivity parameters $a_{ti}$ were controlled by the interaction between an environmental signal and the stream-level population environmental-tolerance curves.

We generated the environmental-tolerance parabolas according to

\begin{equation}
  a_{ti} =
  \begin{cases}
    W_i (e_t - e_i^{\mathrm{opt}})^2 + a_i^{\mathrm{max}},
      & \text{if } a_{ti} > 0\\
      0, & \text{if } a_{ti} \leq 0
  \end{cases}
\end{equation}

\noindent where $W_i$ controls the width of the curve for population $i$, $e_t$ represents the environmental value at generation $t$, $e_i^{\mathrm{opt}}$ represents the optimal environmental value for population $i$, and $a_i^{\mathrm{max}}$ represents the maximum possible $a$ value for population $i$. We set the $W_i$ parameters (Table \ref{t:pars}) and calculated the $a_i^{\mathrm{max}}$ parameters so that the area under each curve $A_i$ was equal. See Figure \ref{f:curves}a for example environmental-tolerance curves.

\subsubsection{Straying}

We implemented straying as in \citet{cooper1999}. We set up the metapopulation in a simple scenario: we arranged the populations in a line and those that were nearer to each other were more likely to stray between each other. Two parameters controlled the straying: the fraction of fish $f_{\mathrm{stray}}$ that stray from their natal stream in any generation and the rate $m$ at which this straying between streams decays with distance. We calculated the number of salmon straying from stream $j$ to stream $i$ as

\begin{equation}
  \mathrm{strays}_{tij} = f_{\mathrm{stray}} R_{tj}
    \frac{e^{-m \lvert i-j \rvert }}
      {\displaystyle\sum\limits_{\substack{k = 1 \\
    k \neq j}}^{n} e^{-m \lvert k-j \rvert }}
  \label{eq:stray}
\end{equation}

\noindent where $R_{tj}$ is the number of returning salmon at generation $t$ whose natal stream was stream $j$. The subscript $k$ represents a stream ID and $n$ the number of populations. The denominator is a normalizing constant to ensure the desired fraction of fish stray. See Figure \ref{f:stray} for an example straying matrix.

\subsection{Fishing}

Our simulation used a simple set of rules to establish escapement targets and harvest the fish. Every $f_\mathrm{assess}$ years (default of five years) our simulation fitted a spawner-return function and target harvest rate $H_{\mathrm{tar}}$ was set based on \citet{hilborn1992} as

\begin{equation}
  H_{\mathrm{tar}} = \frac{A}{b (0.5 - 0.07a)}
  \label{eq:esc}
\end{equation}

\noindent where $A$ represents the return abundance and $a$ and $b$ represent the Ricker model parameters. We included implementation uncertainty in the actual harvest rate $H_{\mathrm{act}}$ as

\begin{equation}
  H_{\mathrm{act}} = \mathrm{beta}(\alpha_h, \beta_h)
\end{equation}

\noindent where $\alpha_h$ and $\beta_h$ are the location and shape parameters in a beta distribution. They can be calculated from the desired mean $\mu_h$ and standard deviation $\sigma_h$ as \citep[p.~97]{morgan1990}

\begin{align}
  \alpha_h &= \mu_h^2
                \left(
                \frac{1 - \mu_h}{\sigma_h^2} - \frac{1}{\mu_h}
                \right)\\
   \beta_h &= \alpha \left({\frac{1}{\mu_h} - 1}\right).
\end{align}

\noindent Further, to establish a range of spawner-return values and to mimic the start of an open-access fishery, for the first 30 years we drew the fraction of fish harvested randomly from a uniform distribution between 0.1 and 0.9. We discarded these initial 30 years as a burn-in period throughout our analyses.

\subsection{Environmental dynamics}

We evaluated portfolio performance under short- and long-term environmental dynamics. We represented short-term dynamics as a stationary first-order autoregressive process, AR(1), with correlation $\rho_e$

\begin{equation}
  e_t = e_{t-1} \rho_e + d_t, d_t \sim \mathrm{N}(0, \sigma_d)
\end{equation}

\noindent where $e_t$ represents the environmental value in generation $t$ and $d$ represents normally distributed deviations of mean 0 and standard deviation $\sigma_d$. We represented long-term environmental dynamics as a linear shift in the environmental value through time

\begin{equation}
  e_t = \beta_e t - \overline{\beta_e t}
\end{equation}

\noindent where $\beta_e$ represents the slope. To maintain a balanced response, we centered the trend by subtracting the mean $\overline{\beta_e t}$ so that midway through the simulation (after any burn-in period) the environmental value was at the mean environmental tolerance.

\subsection{Conservation rules of thumb}

We evaluated two sets of conservation rules of thumb: (1) spatial response diversity conservation strategies in an idealistic scenario where you can detect response diversity, and (2) a more realistic scenario where we know little about response diversity and we're left with a choice of how many populations to conserve.

We evaluated four spatial conservation rules of thumb (Figure \ref{f:curves}b--e). In all spatial scenarios, we conserved four populations and set the unfished equilibrium biomass of the remaining populations to near elimination (five salmon). These reduced populations could still receive straying salmon but were unlikely to rebuild on their own to a substantial abundance. The four scenarios were:

\begin{enumerate}
\def\labelenumi{\arabic{enumi}.}
\itemsep1pt\parskip0pt\parsep0pt
\item
  Conserve an even sampling of response diversity.
\item
  Conserve the most stable populations only.
\item
  Conserve one half of the metapopulation.
\item
  Conserve the other half of the metapopulation.
\end{enumerate}

In reality we rarely know precise levels of response diversity. We therefore additionally considered a case where the conservation was randomly assigned with respect to response diversity but where different numbers of streams could be conserved. We considered conserving from two to 16 streams. Similarly to the spatial strategies, we reduced the capacity of the remaining streams to the nominal level of five salmon.

\section{Results}

\subsection{Which populations to conserve?}

\subsubsection{Short-term environment}

Given strong short-term environmental fluctuations, conserving response diversity buffers the risk properties of an ecological portfolio (Fig.~\ref{f:sp-mv}a). In our simulation, the median variance of generation-to-generation rate of change in abundance was X times lower given balanced response diversity (full range of responses or most stable only vs.~conserving one half or the other). In fact, even though by conserving the full range of responses, the portfolio was comprised of warm and cool-thriving populations that were more variable on their own, each was balanced by an opposing population. The portfolio risk was therefore comparable between the full range of responses and most stable only portfolios.

We can see the mechanism behind these portfolio properties by inspecting example population time series (Fig.~\ref{f:sp-mv}c, d). If only the upper or lower half of response diversity is conserved, the portfolio tends to do well or poorly depending on the environmental conditions (Fig.~\ref{f:sp-mv}d). This risk is buffered with balanced response diversity (Fig.~\ref{f:sp-mv}c).

\subsubsection{Long-term environment}

Given long-term environmental change, the choice of which populations to conserve affects the return properties of an ecological portfolio (Fig.~\ref{f:sp-mv}b). By conserving balance response diversity, an ecological manager is hedging his or her bets on what will happen with the environment and how the populations will respond. The typical return for a balanced response diversity strategy was zero --- the metapopulation neither increased or decreased in abundance in the long run. By conserving only the upper or lower half of response diversity, a conservation manager is putting all his or her eggs in one basket --- the metapopulation might do really well through time or it might do really poorly. The example metapopulation abundance time series (Fig.~\ref{f:sp-mv}d, f) illustrate this effect. By conserving response diversity, when one population is doing poorly, another is doing well and the metapopulation abundance remains stationary through time.

Notably, in theses simulations, if a managers invested in the populations that were doing well at the beginning they would have had the lowest rate-of-return portfolio in the end (purple portfolios in Fig.~\ref{f:sp-mv}b).

Spatial conservation strategies in the face of longterm environmental change hinge on whether you ``get it right'' --- whether you choose just the right populations to conserve.

\subsection{How many populations to conserve?}

Given a scenario where we don't know the distribution of population-level response diversity, portfolio optimization informs us about the risk buffering from maintaining multiple populations (Fig.~\ref{f:n-mv}). In short, investing in more populations buffers portfolio risk.

\subsubsection{Short-term environment}

Given short-term environmental noise, conserving more populations buffers portfolio risk while the random conservation of response diversity creates a spread of metapopulation risk for the same number of populations conserved (Fig.~\ref{f:n-mv}a). For example, a metapopulation with eight conserved populations is X times less risky than a metapopulation with only four. We can see this risk-buffering effect through example metapopulations in Fig.~\ref{f:n-mv}c and \ref{f:n-mv}d. We note that the risk-return axes of portfolio optimization ignore the absolute-abundance dimension (Fig.~\ref{f:n-mv}d). As one would expect, conserving fewer populations also results in lower-abundance metapopulations.

\subsubsection{Long-term environment}

Given long-term environmental noise, conserving more populations also buffers portfolio risk. However, in contrast to the short-term environmental noise scenario, the unknown response diversity creates a spread of possible metapopulation return for the same number of conserved populations (Fig.~\ref{f:n-mv}b). Here, the number of populations conserved buffers non-systematic (i.e.~not-environmentally-driven) stochasticity.

\section{Discussion}

\section{Acknowledgements}

\bibliographystyle{apalike}

\bibliography{jshort,ms}

\clearpage

\section{Tables}

\begin{table}[h!]
\centering
\small
\caption{Components of salmon metapopulation portfolios KEEP THIS?}
\begin{tabular}{p{3.6cm}p{7.5cm}}
\toprule
Component          & Definition for the salmon portfolio\\
\midrule
Assets             & Stream-level salmon populations; possibly a Viable Salmonid Population\\
Portfolio          & The salmon metapopulation; possibly an Evolutionarily Significant Unit\\
Portfolio managers & Salmon managers\\
Investors          & Salmon managers, conservation agency, or salmon fishers\\
Asset weights      & Carrying capacity (specifically the $b$ parameters in a Ricker model)\\
Asset returns      & Rate of change of generation-to-generation salmon metapopulation abundance\\
Asset risk         & Variance of generation-to-generation salmon metapopulation abundance\\
\bottomrule
\end{tabular}
\label{t:port}
\end{table}

\clearpage

\begin{table}[h!]
\centering
\footnotesize
\caption{Input parameters to the salmon metapopulation simulation with default values.}
\begin{tabular}{p{7.7cm}p{1.4cm}p{3.6cm}}
\toprule
Description                                                          & Symbol                & Value \\
\midrule

\textit{Population dynamics parameters}                              &                       & \\
Stock-recruit residual standard deviation (on log scale)             & $\sigma_r$            & 0.30  \\
AR(1) serial correlation of stock-recruit residuals                  & $\rho_w$              & 0.40  \\
Fraction of fish that stray from natal streams                       & $f_{\mathrm{stray}}$  & 0.02  \\
Exponential rate of decay of straying with distance                  & $m$                   & 0.3  \\

\noalign{\vskip 3mm}
\textit{Environmental parameters}                                    &                       & \\
Width of environmental-tolerance curves for populations $i$ 1 to $n$ & $W_i$                 & seq(0.05, 0.02, 0.02 0.05, length = n pop)\\
Optimum environmental value for populations $i$ 1 to $n$             & $e_i^{\mathrm{opt}}$  & seq(13, 19, length = n pop)\\
Area under each environmental-tolerance curve in environmental units & $A$                   & 30\\

Standard deviation of short-term environmental fluctuations          & $\sigma_d$            & 2 \\
AR(1) autocorrelation of short-term environmental fluctuations       & $\rho_e$              & 0.1 \\
Slope of long-term environmental signal                              & $\beta_e$             & 0.114 \\

\noalign{\vskip 3mm}
\textit{Fishery parameters}                                          &                       & \\
Standard deviation of beta distribution for implementation error     & $\sigma_{h}$          & 0.05  \\
Frequency of assessment (years)                                      & $f_{\mathrm{assess}}$ & 5  \\
\bottomrule
\end{tabular}
\label{t:pars}
\end{table}

\clearpage

\section{Figures}

\clearpage

\begin{figure}[htbp]
\centering
\includegraphics[height=5.5in]{../examples/simulation-diagram2.pdf}
\caption{Flow chart of the salmon-metapopulation simulation. There are $n$ salmon populations and $t$ generations. Blue text indicates values that are generated before the simulation progresses through time. Red text indicates steps in which calculations are performed through time. Black text indicates values that are calculated. Grey text indicates parameters that can be set. Green text indicates the looping structure of the simulation.}
\label{f:sim-flow}
\end{figure}

\clearpage

\begin{figure}[htbp]
\centering
\includegraphics[width=3.0in]{../examples/thermal-curve-scenarios.pdf}
\caption{Different ways of prioritizing response-diversity conservation. Panel a shows the thermal tolerance cures for ten possible populations and panels b--e show different ways of prioritizing four of those populations. The curves describe how productivity varies with the environment for a given population. Some populations thrive at low environmental values (cool colours) and some at high (warm colours) values. Some are tolerant to a wider range of environmental conditions (yellow-to-green colours) but with a lower maximum productivity. The total possible productivity (the area under the curves) is the same for each population.}
\label{f:curves}
\end{figure}

\clearpage

\begin{figure}[htbp]
\centering
\includegraphics[width=4.0in]{../examples/spatial-arma-sim.pdf}
\caption{The components of an example metapopulation simulation.  We show, from top to bottom, the environmental signal, the resulting productivity parameter (Ricker $a$), the salmon returns, fisheries catch, salmon escapement, salmon straying from their natal streams, salmon joining from other streams, stock-recruit residuals on a log scale, and the estimated $a$ and $b$ parameters in the fitted Ricker curve. The colored lines indicate populations that thrive at low (cool colours) to high (warm colours) environmental values.}
\label{f:sp-eg}
\end{figure}

\clearpage

\begin{figure}[htbp]
\centering
\includegraphics[width=5.0in]{../examples/spatial-mv.pdf}
\caption{The importance of preserving environmental response diversity through spatial conservation strategies. The conservation strategies correspond to Fig.~\ref{f:curves} and represent conserving a range of responses (green), the most stable populations only (orange), or one type of environmental response (purple and pink).  In risk-return space we show environmental scenarios that are comprised primarily of (a) short-term and (b) long-term environmental fluctuations (see Fig.~X). The dots show simulated metapopulations and the contours show 25\% and 75\% quantiles across 500 simulations per strategy. We also show example metapopulation abundance time series for the (c, e) short-term and (d, f) long-term  environmental-fluctuation scenarios.}
\label{f:sp-mv}
\end{figure}

\clearpage

\begin{figure}[htbp]
\centering
\includegraphics[width=5.0in]{../examples/cons-plans-n.pdf}
\caption{The importance of preserving as many subpopulations as possible when we don't know how response diversity is distributed. In risk-return space we show environmental scenarios that are comprised primarily of (a) short-term and (b) long-term environmental fluctuations (see Fig.~X). We show metapopulations in which 2 (red), 4 (orange), 8 (yellow), or 16 (green) populations of random response diversity are conserved. The dots show simulated metapopulations and the contours show 25\% and 75\% quantiles across 500 simulations per strategy. We also show example metapopulation (c) rate-of-change and (d) abundance time series for the short-term environmental-fluctuation scenario.}
\label{f:n-mv}
\end{figure}

\clearpage


\bibliographystyle{ecology3}
\bibliography{jshort,ms}
%\documentclass[12pt]{article}
\usepackage{geometry}
\geometry{verbose,letterpaper,tmargin=2.54cm,bmargin=2.54cm,lmargin=2.54cm,rmargin=2.54cm} 
\geometry{letterpaper}
\usepackage{amssymb}
\usepackage{amsmath}
\usepackage{amsfonts}
\usepackage{graphicx}
\usepackage{setspace}
\usepackage[round]{natbib}
\bibpunct{(}{)}{;}{a}{}{;}
\textheight 22.0cm
\usepackage{lineno}
\usepackage{xcolor}
\renewcommand\linenumberfont{\normalfont\tiny\sffamily\color{gray}}
\usepackage{booktabs}
\usepackage{cite}

\newlabel{f:flowchart}{{1}{999}}
\newlabel{f:curves}{{2}{999}}
\newlabel{f:ts}{{3}{999}}
\newlabel{f:sp}{{4}{999}}
\newlabel{f:n}{{5}{999}}

\newcommand{\somR}{Appendix A}
\newcommand{\somparam}{Appendix B}
\newcommand{\somstray}{Appendix C}
\newcommand{\somsens}{Appendix D}
\newcommand{\somcor}{Appendix E}
\newcommand{\somts}{Appendix F}

% SOM:
\newlabel{t:pars}{{B1}{2}}
\newlabel{f:stray}{{B1}{3}}
\newlabel{f:eg-sens}{{D1}{4}}
\newlabel{f:ret-corr}{{E1}{5}}
\newlabel{f:eg-sp-arma-full}{{F1}{6}}
\newlabel{f:eg-sp-arma-half}{{F2}{7}}
\newlabel{f:eg-sp-linear-full}{{F3}{8}}
\newlabel{f:eg-sp-linear-half}{{F4}{9}}
\newlabel{f:eg-n-arma-two}{{F5}{10}}
\newlabel{f:eg-n-arma-sixteen}{{F6}{11}}
\newlabel{f:eg-n-linear-two}{{F7}{12}}
\newlabel{f:eg-n-linear-sixteen}{{F8}{13}}


\DeclareGraphicsExtensions{.pdf}

\widowpenalty=10000
\clubpenalty=10000

%\usepackage{titlesec}
%\titlespacing\section{0pt}{6pt plus 4pt minus 2pt}{-8pt plus 2pt minus 2pt}
%\titlespacing\subsection{0pt}{6pt plus 4pt minus 2pt}{-8pt plus 2pt minus 2pt}

\title{Portfolio conservation of metapopulations under\\climate change}

\author{
Sean C. Anderson$^{1\ast}$ \and Jonathan W. Moore$^{1,2}$ \and Michelle M. McClure$^3$ \and
Nicholas K. Dulvy$^1$ \and Andrew B. Cooper$^2$
}
\date{}

% remove numbers in front of sections:
\makeatletter
\renewcommand\@seccntformat[1]{}
\makeatother

\hyphenation{meta-pop-ulation meta-pop-ulations sub-pop-ulations sub-pop-ulation e-con-o-mist en-vi-ron-men-tal pri-or-i-ti-za-tion}

\begin{document}
\raggedright


\linenumbers

%\modulolinenumbers[2]
\begin{spacing}{1.9}
\setlength{\parindent}{0.9cm}

\maketitle

\input{ms}

\bibliographystyle{ecology3}
\bibliography{jshort,ms}
%\input{anderson-etal-metafolio.bbl}

\clearpage

\section{Supplemental Material}

\noindent
\textsc{Appendix A.} The \texttt{metafolio} \textsf{R} package.

\noindent
\textsc{Appendix B.} Simulation input parameters and default values.

\noindent
\textsc{Appendix C.} An example straying matrix.

\noindent
\textsc{Appendix D.} Sensitivity illustration with alternative parameter values.

\noindent
\textsc{Appendix E.} An illustration of the correlation between populations.

\noindent
\textsc{Appendix F.} Example simulated time series from alternative conservation scenarios.

\clearpage

\section{Figure legends}

%\begin{center}
%\end{center}
\textsc{Fig. 1}. Flow chart of the salmon-metapopulation simulation. There are $n$ salmon populations and $t$ generations. Blue text indicates values that are generated before the simulation progresses through time. Red text indicates steps in which calculations are performed through time. Black text indicates values that are calculated. Grey text indicates parameters that can be set. Green text indicates the looping structure of the simulation.

%\clearpage

%\begin{center}
%\includegraphics[width=2.9in]{../examples/thermal-curve-scenarios}
%\end{center}
\bigskip
\noindent
\textsc{Fig. 2}. Different ways of prioritizing response-diversity conservation. Panel a shows thermal tolerance curves for ten possible populations and panels b--e show different ways of prioritizing four of those populations. The curves describe how productivity varies with temperature for a given population. Some populations thrive at low temperatures (cool colours) and some at warm temperatures (warm colours). Some are tolerant to a wider range of environmental conditions (yellow-to-green colours) but with a lower maximum productivity. The total possible productivity (the area under the curves) is the same for each population.

%\clearpage

%\begin{center}
%\includegraphics[width=4.0in]{../examples/spatial-arma-sim-full}
%\end{center}
\bigskip
\noindent
\textsc{Fig. \ref{f:ts}}. The components of an example metapopulation simulation. We show, from top to bottom, the temperature signal, the resulting productivity parameter (Ricker $a$), the salmon returns, fisheries catch, salmon escapement, salmon straying from their natal streams, salmon joining from other streams, spawner-return residuals on a log scale, and the estimated $a$ and $b$ parameters in the fitted Ricker curve. The colored lines indicate populations that thrive at low (cool colours) to high (warm colours) temperatures.

%\clearpage

%\begin{center}
%\includegraphics[width=4.5in]{../examples/spatial-mv}
%\end{center}
\bigskip
\noindent
\textsc{Fig. 4}. The importance of preserving environmental response diversity through spatial conservation strategies. The conservation strategies correspond to figure 2 and represent conserving a range of responses (green), the most stable populations only (orange), or one type of environmental response (purple and pink). In risk-return space we show environmental scenarios that are comprised primarily of (a) short-term and (b) long-term environmental fluctuations. The dots show simulated metapopulations and the contours show 25\% and 75\% quantiles across 500 simulations per strategy. We also show example metapopulation abundance time series for the (c, e) short-term and (d, f) long-term environmental-fluctuation scenarios. The grey line (a, b) indicates the efficient frontier across all simulated metapopulations --- metapopulations with the minimum variability for a given level of growth rate.

%\clearpage

%\begin{center}
%\includegraphics[width=4.5in]{../examples/cons-plans-n}
%\end{center}
\bigskip
\noindent
\textsc{Fig. 5}. The importance of preserving as many populations as possible when we don't know how response diversity is distributed. In risk-return space we show environmental scenarios that are comprised primarily of (a) short-term and (b) long-term environmental fluctuations. We show metapopulations in which 2 (red), 4 (orange), 8 (yellow), or 16 (green) populations of random response diversity are conserved. The dots show simulated metapopulations and the contours show 25\% and 75\% quantiles across 500 simulations per strategy. We also show example metapopulation (c) rate-of-change and (d) abundance time series for the short-term environmental-fluctuation scenario. The grey line (a, b) indicates the efficient frontier across all simulated metapopulations --- metapopulations with the minimum variability for a given level of growth rate.

\clearpage

\section{Figures}

\begin{center}
\includegraphics[height=5.5in]{../examples/simulation-diagram3}\\
\textsc{Fig.} 1
\clearpage
\includegraphics[width=2.9in]{../examples/thermal-curve-scenarios}\\
\textsc{Fig.} 2
\clearpage
\includegraphics[width=4.0in]{../examples/spatial-arma-sim-full}\\
\textsc{Fig.} 3
\clearpage
\includegraphics[width=4.5in]{../examples/spatial-mv}\\
\textsc{Fig.} 4
\clearpage
\includegraphics[width=4.5in]{../examples/cons-plans-n}\\
\textsc{Fig.} 5
\clearpage
\end{center}

\end{spacing}

%\setlength{\parskip}{8pt}
%\setlength{\parindent}{0cm}

 \begin{spacing}{1.1}

 \setcounter{page}{1}
 \nolinenumbers
 \include{som}

 \end{spacing}

\end{document}


\clearpage

\section{Supplemental Material}

\noindent
\textsc{Appendix A.} The \texttt{metafolio} \textsf{R} package.

\noindent
\textsc{Appendix B.} Simulation input parameters and default values.

\noindent
\textsc{Appendix C.} An example straying matrix.

\noindent
\textsc{Appendix D.} Sensitivity illustration with alternative parameter values.

\noindent
\textsc{Appendix E.} An illustration of the correlation between populations.

\noindent
\textsc{Appendix F.} Example simulated time series from alternative conservation scenarios.

\clearpage

\section{Figure legends}

%\begin{center}
%\end{center}
\textsc{Fig. 1}. Flow chart of the salmon-metapopulation simulation. There are $n$ salmon populations and $t$ generations. Blue text indicates values that are generated before the simulation progresses through time. Red text indicates steps in which calculations are performed through time. Black text indicates values that are calculated. Grey text indicates parameters that can be set. Green text indicates the looping structure of the simulation.

%\clearpage

%\begin{center}
%\includegraphics[width=2.9in]{../examples/thermal-curve-scenarios}
%\end{center}
\bigskip
\noindent
\textsc{Fig. 2}. Different ways of prioritizing response-diversity conservation. Panel a shows thermal tolerance curves for ten possible populations and panels b--e show different ways of prioritizing four of those populations. The curves describe how productivity varies with temperature for a given population. Some populations thrive at low temperatures (cool colours) and some at warm temperatures (warm colours). Some are tolerant to a wider range of environmental conditions (yellow-to-green colours) but with a lower maximum productivity. The total possible productivity (the area under the curves) is the same for each population.

%\clearpage

%\begin{center}
%\includegraphics[width=4.0in]{../examples/spatial-arma-sim-full}
%\end{center}
\bigskip
\noindent
\textsc{Fig. \ref{f:ts}}. The components of an example metapopulation simulation. We show, from top to bottom, the temperature signal, the resulting productivity parameter (Ricker $a$), the salmon returns, fisheries catch, salmon escapement, salmon straying from their natal streams, salmon joining from other streams, spawner-return residuals on a log scale, and the estimated $a$ and $b$ parameters in the fitted Ricker curve. The colored lines indicate populations that thrive at low (cool colours) to high (warm colours) temperatures.

%\clearpage

%\begin{center}
%\includegraphics[width=4.5in]{../examples/spatial-mv}
%\end{center}
\bigskip
\noindent
\textsc{Fig. 4}. The importance of preserving environmental response diversity through spatial conservation strategies. The conservation strategies correspond to figure 2 and represent conserving a range of responses (green), the most stable populations only (orange), or one type of environmental response (purple and pink). In risk-return space we show environmental scenarios that are comprised primarily of (a) short-term and (b) long-term environmental fluctuations. The dots show simulated metapopulations and the contours show 25\% and 75\% quantiles across 500 simulations per strategy. We also show example metapopulation abundance time series for the (c, e) short-term and (d, f) long-term environmental-fluctuation scenarios. The grey line (a, b) indicates the efficient frontier across all simulated metapopulations --- metapopulations with the minimum variability for a given level of growth rate.

%\clearpage

%\begin{center}
%\includegraphics[width=4.5in]{../examples/cons-plans-n}
%\end{center}
\bigskip
\noindent
\textsc{Fig. 5}. The importance of preserving as many populations as possible when we don't know how response diversity is distributed. In risk-return space we show environmental scenarios that are comprised primarily of (a) short-term and (b) long-term environmental fluctuations. We show metapopulations in which 2 (red), 4 (orange), 8 (yellow), or 16 (green) populations of random response diversity are conserved. The dots show simulated metapopulations and the contours show 25\% and 75\% quantiles across 500 simulations per strategy. We also show example metapopulation (c) rate-of-change and (d) abundance time series for the short-term environmental-fluctuation scenario. The grey line (a, b) indicates the efficient frontier across all simulated metapopulations --- metapopulations with the minimum variability for a given level of growth rate.

\clearpage

\section{Figures}

\begin{center}
\includegraphics[height=5.5in]{../examples/simulation-diagram3}\\
\textsc{Fig.} 1
\clearpage
\includegraphics[width=2.9in]{../examples/thermal-curve-scenarios}\\
\textsc{Fig.} 2
\clearpage
\includegraphics[width=4.0in]{../examples/spatial-arma-sim-full}\\
\textsc{Fig.} 3
\clearpage
\includegraphics[width=4.5in]{../examples/spatial-mv}\\
\textsc{Fig.} 4
\clearpage
\includegraphics[width=4.5in]{../examples/cons-plans-n}\\
\textsc{Fig.} 5
\clearpage
\end{center}

\end{spacing}

%\setlength{\parskip}{8pt}
%\setlength{\parindent}{0cm}

 \begin{spacing}{1.1}

 \setcounter{page}{1}
 \nolinenumbers
 \section{Supplementary Information}

\subsection{Supplementary code}

The \texttt{metafolio} \texttt{R} package and documentation. Some
details on what you can do with the package.

\subsection{Supplementary figures}

\begin{figure}[htbp]
\centering
\includegraphics[width=4.0in]{../examples/figure/plot-various-options-ts-3pops.pdf}
\caption{The impact of increasing or decreasing various parameter values in the simulations. The different lines represent different salmon populations. (NEED TO ADD PARAMETER VALUES AND EXPAND THIS SLIGHTLY)}
\label{f:eg-sens}
\end{figure}

\clearpage

\begin{figure}[htbp]
\centering
\includegraphics[width=4.0in]{../examples/figure/stray-matrix.pdf}
\caption{An example straying matrix. The rows and columns represent different 
populations (indicated by population number). Dark blue indicates a high rate 
of straying and light blue indicates a low rate of straying.}
\label{f:stray}
\end{figure}

\clearpage

\begin{figure}[htbp]
\centering
\includegraphics[width=4.5in]{../examples/spatial-arma-sim.pdf}
\caption{Spatial and short-term environmental fluctuations}
\label{f:eg-sp-arma}
\end{figure}

\clearpage

\begin{figure}[htbp]
\centering
\includegraphics[width=4.5in]{../examples/spatial-linear-sim.pdf}
\caption{Spatial and long-term environmental fluctuations}
\label{f:eg-sp-linear}
\end{figure}

\clearpage

\begin{figure}[htbp]
\centering
\includegraphics[width=4.5in]{../examples/n-arma-sim.pdf}
\caption{Number and short-term environmental fluctuations}
\label{f:eg-n-arma}
\end{figure}

\clearpage

\begin{figure}[htbp]
\centering
\includegraphics[width=4.5in]{../examples/n-linear-sim.pdf}
\caption{Number and long-term environmental change}
\label{f:eg-n-linear}
\end{figure}

\clearpage


 \end{spacing}

\end{document}


\clearpage

\section{Supplemental Material}

\noindent
\textsc{Appendix A.} The \texttt{metafolio} \textsf{R} package.

\noindent
\textsc{Appendix B.} Simulation input parameters and default values.

\noindent
\textsc{Appendix C.} An example straying matrix.

\noindent
\textsc{Appendix D.} Sensitivity illustration with alternative parameter values.

\noindent
\textsc{Appendix E.} An illustration of the correlation between populations.

\noindent
\textsc{Appendix F.} Example simulated time series from alternative conservation scenarios.

\clearpage

\section{Figure legends}

%\begin{center}
%\end{center}
\textsc{Fig. 1}. Flow chart of the salmon-metapopulation simulation. There are $n$ salmon populations and $t$ generations. Blue text indicates values that are generated before the simulation progresses through time. Red text indicates steps in which calculations are performed through time. Black text indicates values that are calculated. Grey text indicates parameters that can be set. Green text indicates the looping structure of the simulation.

%\clearpage

%\begin{center}
%\includegraphics[width=2.9in]{../examples/thermal-curve-scenarios}
%\end{center}
\bigskip
\noindent
\textsc{Fig. 2}. Different ways of prioritizing response-diversity conservation. Panel a shows thermal tolerance curves for ten possible populations and panels b--e show different ways of prioritizing four of those populations. The curves describe how productivity varies with temperature for a given population. Some populations thrive at low temperatures (cool colours) and some at warm temperatures (warm colours). Some are tolerant to a wider range of environmental conditions (yellow-to-green colours) but with a lower maximum productivity. The total possible productivity (the area under the curves) is the same for each population.

%\clearpage

%\begin{center}
%\includegraphics[width=4.0in]{../examples/spatial-arma-sim-full}
%\end{center}
\bigskip
\noindent
\textsc{Fig. \ref{f:ts}}. The components of an example metapopulation simulation. We show, from top to bottom, the temperature signal, the resulting productivity parameter (Ricker $a$), the salmon returns, fisheries catch, salmon escapement, salmon straying from their natal streams, salmon joining from other streams, spawner-return residuals on a log scale, and the estimated $a$ and $b$ parameters in the fitted Ricker curve. The colored lines indicate populations that thrive at low (cool colours) to high (warm colours) temperatures.

%\clearpage

%\begin{center}
%\includegraphics[width=4.5in]{../examples/spatial-mv}
%\end{center}
\bigskip
\noindent
\textsc{Fig. 4}. The importance of preserving environmental response diversity through spatial conservation strategies. The conservation strategies correspond to figure 2 and represent conserving a range of responses (green), the most stable populations only (orange), or one type of environmental response (purple and pink). In risk-return space we show environmental scenarios that are comprised primarily of (a) short-term and (b) long-term environmental fluctuations. The dots show simulated metapopulations and the contours show 25\% and 75\% quantiles across 500 simulations per strategy. We also show example metapopulation abundance time series for the (c, e) short-term and (d, f) long-term environmental-fluctuation scenarios. The grey line (a, b) indicates the efficient frontier across all simulated metapopulations --- metapopulations with the minimum variability for a given level of growth rate.

%\clearpage

%\begin{center}
%\includegraphics[width=4.5in]{../examples/cons-plans-n}
%\end{center}
\bigskip
\noindent
\textsc{Fig. 5}. The importance of preserving as many populations as possible when we don't know how response diversity is distributed. In risk-return space we show environmental scenarios that are comprised primarily of (a) short-term and (b) long-term environmental fluctuations. We show metapopulations in which 2 (red), 4 (orange), 8 (yellow), or 16 (green) populations of random response diversity are conserved. The dots show simulated metapopulations and the contours show 25\% and 75\% quantiles across 500 simulations per strategy. We also show example metapopulation (c) rate-of-change and (d) abundance time series for the short-term environmental-fluctuation scenario. The grey line (a, b) indicates the efficient frontier across all simulated metapopulations --- metapopulations with the minimum variability for a given level of growth rate.

\clearpage

\section{Figures}

\begin{center}
\includegraphics[height=5.5in]{../examples/simulation-diagram3}\\
\textsc{Fig.} 1
\clearpage
\includegraphics[width=2.9in]{../examples/thermal-curve-scenarios}\\
\textsc{Fig.} 2
\clearpage
\includegraphics[width=4.0in]{../examples/spatial-arma-sim-full}\\
\textsc{Fig.} 3
\clearpage
\includegraphics[width=4.5in]{../examples/spatial-mv}\\
\textsc{Fig.} 4
\clearpage
\includegraphics[width=4.5in]{../examples/cons-plans-n}\\
\textsc{Fig.} 5
\clearpage
\end{center}

\end{spacing}

%\setlength{\parskip}{8pt}
%\setlength{\parindent}{0cm}

 \begin{spacing}{1.1}

 \setcounter{page}{1}
 \nolinenumbers
 \section{Supplementary Information}

\subsection{Supplementary code}

The \texttt{metafolio} \texttt{R} package and documentation. Some
details on what you can do with the package.

\subsection{Supplementary figures}

\begin{figure}[htbp]
\centering
\includegraphics[width=4.0in]{../examples/figure/plot-various-options-ts-3pops.pdf}
\caption{The impact of increasing or decreasing various parameter values in the simulations. The different lines represent different salmon populations. (NEED TO ADD PARAMETER VALUES AND EXPAND THIS SLIGHTLY)}
\label{f:eg-sens}
\end{figure}

\clearpage

\begin{figure}[htbp]
\centering
\includegraphics[width=4.0in]{../examples/figure/stray-matrix.pdf}
\caption{An example straying matrix. The rows and columns represent different 
populations (indicated by population number). Dark blue indicates a high rate 
of straying and light blue indicates a low rate of straying.}
\label{f:stray}
\end{figure}

\clearpage

\begin{figure}[htbp]
\centering
\includegraphics[width=4.5in]{../examples/spatial-arma-sim.pdf}
\caption{Spatial and short-term environmental fluctuations}
\label{f:eg-sp-arma}
\end{figure}

\clearpage

\begin{figure}[htbp]
\centering
\includegraphics[width=4.5in]{../examples/spatial-linear-sim.pdf}
\caption{Spatial and long-term environmental fluctuations}
\label{f:eg-sp-linear}
\end{figure}

\clearpage

\begin{figure}[htbp]
\centering
\includegraphics[width=4.5in]{../examples/n-arma-sim.pdf}
\caption{Number and short-term environmental fluctuations}
\label{f:eg-n-arma}
\end{figure}

\clearpage

\begin{figure}[htbp]
\centering
\includegraphics[width=4.5in]{../examples/n-linear-sim.pdf}
\caption{Number and long-term environmental change}
\label{f:eg-n-linear}
\end{figure}

\clearpage


 \end{spacing}

\end{document}


\clearpage

\section{Figure legends}

%\begin{center}
%\includegraphics[height=5.5in]{../examples/simulation-diagram3}
%\end{center}
%\textbf{Figure 1.} Flow chart of the salmon-metapopulation simulation. There are $n$ salmon populations and $t$ generations. Blue text indicates values that are generated before the simulation progresses through time. Red text indicates steps in which calculations are performed through time. Black text indicates values that are calculated. Grey text indicates parameters that can be set. Green text indicates the looping structure of the simulation.

%\clearpage

%\begin{center}
%\includegraphics[width=2.9in]{../examples/thermal-curve-scenarios}
%\end{center}
\textbf{Figure 1.} Different ways of prioritizing response-diversity conservation. Panel a shows thermal tolerance curves for ten possible populations and panels b--e show different ways of prioritizing four of those populations. The curves describe how productivity varies with temperature for a given population. Some populations thrive at low temperatures (cool colours) and some at warm temperatures (warm colours). Some are tolerant to a wider range of environmental conditions (yellow-to-green colours) but with a lower maximum productivity. The total possible productivity (the area under the curves) is the same for each population.

%\clearpage

%\begin{center}
%\includegraphics[width=4.0in]{../examples/spatial-arma-sim-full}
%\end{center}
%\textbf{Figure 3.} The components of an example metapopulation simulation. We show, from top to bottom, the temperature signal, the resulting productivity parameter (Ricker $a$), the salmon returns, fisheries catch, salmon escapement, salmon straying from their natal streams, salmon joining from other streams, spawner-return residuals on a log scale, and the estimated $a$ and $b$ parameters in the fitted Ricker curve. The colored lines indicate populations that thrive at low (cool colours) to high (warm colours) temperatures.

%\clearpage

%\begin{center}
%\includegraphics[width=4.5in]{../examples/spatial-mv}
%\end{center}
\textbf{Figure 2.} The importance of preserving environmental response diversity through spatial conservation strategies. The conservation strategies correspond to figure 2 and represent conserving a range of responses (green), the most stable populations only (orange), or one type of environmental response (purple and pink). In risk-return space we show environmental scenarios that are comprised primarily of (a) short-term and (b) long-term environmental fluctuations. The dots show simulated metapopulations and the contours show 25\% and 75\% quantiles across 500 simulations per strategy. We also show example metapopulation abundance time series for the (c, e) short-term and (d, f) long-term environmental-fluctuation scenarios. The grey line (a, b) indicates the efficient frontier across all simulated metapopulations --- metapopulations with the minimum variability for a given level of growth rate.

%\clearpage

%\begin{center}
%\includegraphics[width=4.5in]{../examples/cons-plans-n}
%\end{center}
\textbf{Figure 3.} The importance of preserving as many populations as possible when we don't know how response diversity is distributed. In risk-return space we show environmental scenarios that are comprised primarily of (a) short-term and (b) long-term environmental fluctuations. We show metapopulations in which 2 (red), 4 (orange), 8 (yellow), or 16 (green) populations of random response diversity are conserved. The dots show simulated metapopulations and the contours show 25\% and 75\% quantiles across 500 simulations per strategy. We also show example metapopulation (c) rate-of-change and (d) abundance time series for the short-term environmental-fluctuation scenario. The grey line (a, b) indicates the efficient frontier across all simulated metapopulations --- metapopulations with the minimum variability for a given level of growth rate.

\clearpage

\end{spacing}

%\setlength{\parskip}{8pt}
%\setlength{\parindent}{0cm}

 %\begin{spacing}{1.1}

 %\setcounter{page}{1}
 %\nolinenumbers
 %\section{Supplementary Information}

\subsection{Supplementary code}

The \texttt{metafolio} \texttt{R} package and documentation. Some
details on what you can do with the package.

\subsection{Supplementary figures}

\begin{figure}[htbp]
\centering
\includegraphics[width=4.0in]{../examples/figure/plot-various-options-ts-3pops.pdf}
\caption{The impact of increasing or decreasing various parameter values in the simulations. The different lines represent different salmon populations. (NEED TO ADD PARAMETER VALUES AND EXPAND THIS SLIGHTLY)}
\label{f:eg-sens}
\end{figure}

\clearpage

\begin{figure}[htbp]
\centering
\includegraphics[width=4.0in]{../examples/figure/stray-matrix.pdf}
\caption{An example straying matrix. The rows and columns represent different 
populations (indicated by population number). Dark blue indicates a high rate 
of straying and light blue indicates a low rate of straying.}
\label{f:stray}
\end{figure}

\clearpage

\begin{figure}[htbp]
\centering
\includegraphics[width=4.5in]{../examples/spatial-arma-sim.pdf}
\caption{Spatial and short-term environmental fluctuations}
\label{f:eg-sp-arma}
\end{figure}

\clearpage

\begin{figure}[htbp]
\centering
\includegraphics[width=4.5in]{../examples/spatial-linear-sim.pdf}
\caption{Spatial and long-term environmental fluctuations}
\label{f:eg-sp-linear}
\end{figure}

\clearpage

\begin{figure}[htbp]
\centering
\includegraphics[width=4.5in]{../examples/n-arma-sim.pdf}
\caption{Number and short-term environmental fluctuations}
\label{f:eg-n-arma}
\end{figure}

\clearpage

\begin{figure}[htbp]
\centering
\includegraphics[width=4.5in]{../examples/n-linear-sim.pdf}
\caption{Number and long-term environmental change}
\label{f:eg-n-linear}
\end{figure}

\clearpage


 %\end{spacing}

\end{document}
